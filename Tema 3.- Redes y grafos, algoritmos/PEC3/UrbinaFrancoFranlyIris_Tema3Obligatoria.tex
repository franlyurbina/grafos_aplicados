\documentclass[12pt,a4paper]{article}

% ----------------------------------------------------
% PAQUETES BÁSICOS
% ----------------------------------------------------
\usepackage[utf8]{inputenc}
\usepackage[T1]{fontenc}
\usepackage[spanish]{babel}
\usepackage{float}
\usepackage{geometry}
\usepackage{graphicx}
\usepackage{booktabs}
\usepackage{fancyhdr}
\usepackage{setspace}
\usepackage{titlesec}
\usepackage{xcolor}
\usepackage{threeparttable}
\usepackage{pgfplots}
\usepackage{pgfplotstable}
\usepackage{caption}
\usepackage{amsmath,amssymb,amsfonts}
\usepackage{enumitem}
\usepackage{multirow}
\usepackage[bottom]{footmisc}
\usepackage{makecell}
\usepackage{tikz}
\usepackage{longtable}
\usepackage{pdflscape}
\usepackage{xcolor,colortbl}
\usepackage{array}
\usepackage{tikz}  
\usetikzlibrary{arrows.meta, positioning}
\usepackage{float}
\usepackage[spanish]{babel}
\usepackage{alltt} 
\usepackage[dvipsnames]{xcolor}
\usepackage[table]{xcolor}
\usepackage{graphicx}
\usetikzlibrary{babel}



\usepackage[hyperfootnotes=true]{hyperref}

\hypersetup{
    colorlinks=true,
    linkcolor=blue!50!black,
    urlcolor=blue!50!black,
    citecolor=blue!50!black
    }
    \usepgfplotslibrary{statistics}
    \pgfplotsset{compat=1.18}
    \usetikzlibrary{calc} 
    \usetikzlibrary{bayesnet}
    \usetikzlibrary{arrows.meta, positioning}
    
    % ----------------------------------------------------
    % FORMATO GENERAL
    % ----------------------------------------------------
    \geometry{margin=2.0cm}
    \setstretch{1.3}
\setlength{\parskip}{0.3em}
\setlength{\parindent}{2em}

\renewcommand{\baselinestretch}{1.25}
\renewcommand{\familydefault}{\rmdefault}

\pagestyle{fancy}
\fancyhf{}
\fancyhead[L]{Máster en Tecnologías del Lenguaje — UNED}
\fancyhead[R]{Minería de Textos}
\fancyfoot[C]{\thepage}

% ----------------------------------------------------
% FORMATO DE SECCIONES
% ----------------------------------------------------
\titleformat{\section}{\Large\bfseries}{\thesection}{0.5em}{}
\titleformat{\subsection}{\large\bfseries}{\thesubsection}{0.5em}{}

% ----------------------------------------------------
% PORTADA
% ----------------------------------------------------
\newcommand{\portada}{
\begin{titlepage}
\centering
\vspace{2cm}

{\Large \textbf{Universidad Nacional de Educación a Distancia (UNED)}}\\[0.5cm]
{\large E.T.S. de Ingeniería Informática — Máster Universitario en Tecnologías del Lenguaje}\\[5cm]
{\huge \textbf{Minería de Textos}}\\[1.5cm]
{\Large \textbf{Reconocimiento de Entidades Nombradas}}\\[0.5cm]
{\Large \textbf{Tarea 3 Obligatoria}}\\[3cm]

\vfill
\begin{flushright}
\textbf{Autor:} Franly Iris Urbina Franco\\
\textbf{DNI:} 71.997.418-N\\
\textbf{Correo:} furbina7@alumno.uned.es\\
\textbf{Fecha:} \today
\end{flushright}
 
{\small Curso 2025-2026 — Tema 3}
\end{titlepage}
}

\begin{document}
\portada








% ----------------------------------------------------
% ÍNDICE
% ----------------------------------------------------
\tableofcontents
\newpage

% ----------------------------------------------------
% INTRODUCCIÓN
% ----------------------------------------------------
\section*{Introducción}
A continuación, se presenta un análisis basado en \texttt{Brown\_10-2.graphML}, un fichero que 
contiene una red léxica de coaparición construida a partir de un corpus textual.

La estructura interna de esta red se organiza en comunidades detectadas 
mediante modularidad, donde cada nodo corresponde a un término y las palabras tienden a agruparse
en campos semánticos diferenciados. Cada arista indica que dos términos aparecen juntos en un
mismo contexto y posee un peso asociado que mide la frecuencia de esa relación.

A partir de esta estructura se generan cincuenta tablas, dos por cada comunidad, con los diez nodos más
relevantes según dos criterios: el grado, que indica cuántas conexiones distintas tiene un término, y el
grado con pesos, que refleja la intensidad total de sus relaciones.

Cada clase modular representa una comunidad de palabras que conforman un vocabulario interno del corpus,
lo que permite identificar grupos temáticos, términos que actúan como centros estructurales y palabras
que funcionan como enlaces dentro de cada dominio semántico. Ordenar por grado permite observar cuántas
palabras diferentes coaparecen con un término al menos una vez, ya que el grado representa el número de
nodos distintos con los que está conectado, identificando así palabras presentes en muchos contextos y
nodos estructuralmente centrales con alta diversidad de relaciones. Por otro lado, el grado ponderado es
la suma de los pesos de todas sus aristas, es decir, cuántas veces en total coaparece ese término con otros,
lo que permite detectar qué palabras funcionan como núcleos temáticos más intensos. De este modo se responde
a dos cuestiones: cuáles son los nodos con más relaciones distintas y cuáles presentan relaciones más fuertes
dentro de cada comunidad.

\begin{table}[h]
\centering
\begin{tabular}{l l}
\hline
Condición & Interpretación \\
\hline
Alto grado + alto peso & Conceptos centrales del tema \\
Alto grado + bajo peso & Términos puente o generales \\
Bajo grado + alto peso & Términos muy específicos \\
Bajo grado + bajo peso & Nodos periféricos \\
\hline
\end{tabular}
\caption{Interpretación estructural de los nodos según grado y grado ponderado}
\end{table}

Los nodos de las tablas representan el significado global de cada comunidad, por lo que el nombre asignado
reflejará su campo semántico mediante la elección de un sustantivo representativo de las etiquetas de los nodos,
con el fin de caracterizar los núcleos modulares por significado dentro del corpus.

Para obtener las comunidades se siguieron las instrucciones: primero se filtró el componente gigante
para eliminar subgrafos irrelevantes; después se calcularon métricas estructurales y se aplicó el
algoritmo de modularidad para detectar comunidades y el algoritmo de distribución \textit{ForceAtlas2}
para la disposición visual del grafo. Esto permite interpretar las comunidades como agrupaciones semánticas
de términos relacionados a partir del análisis de redes léxicas orientado a extraer conocimiento estructural
del corpus.

 
\newpage


\section*{Comunidad 0 - Election}
En la Comunidad 0 se observa una estructura jerárquica definida donde los términos 
\textit{vote} y \textit{election} actúan como centro semántico de la red. Estos nodos 
mantienen las dos primeras posiciones en ambas métricas, lo que demuestra que no solo 
poseen una alta conectividad estructural (grado), sino que sus relaciones son las más 
intensas en términos de coocurrencia recurrente (grado con pesos).
\vspace{0.5cm}

\noindent
\begin{minipage}[t]{0.48\textwidth}
\centering
\begin{tabular}{r l r r}
\hline
Id & Label & Grado & \shortstack{Grado \\ con pesos} \\
\hline
267 & \textcolor{ForestGreen}{vote} & 51 & 74.1021 \\
4 & \textcolor{ForestGreen}{election} & 41 & 61.9433 \\
135 & taxpayer & 38 & 30.4639 \\
100 & \textcolor{ForestGreen}{federal} & 35 & 31.6903 \\
64 & \textcolor{ForestGreen}{government} & 33 & 39.0482 \\
692 & method & 32 & 18.0729 \\
352 & \textcolor{ForestGreen}{tax} & 30 & 34.9038 \\
33 & \textcolor{ForestGreen}{voter} & 29 & 39.3848 \\
266 & \textcolor{ForestGreen}{force} & 28 & 33.8163 \\
22 & report & 26 & 30.0436 \\
\hline
\end{tabular}

\vspace{0.2cm}
\textbf{Top 10 por Grado}
\end{minipage}
\hspace{0.5cm}
\begin{minipage}[t]{0.48\textwidth}
\centering
\begin{tabular}{r l r r}
\hline
Id & Label & Grado & \shortstack{Grado \\ con pesos} \\
\hline
267 & \textcolor{ForestGreen}{vote} & 51 & 74.1021 \\
4 & \textcolor{ForestGreen}{election} & 41 & 61.9433 \\
33 & \textcolor{ForestGreen}{voter} & 29 & 39.3848 \\
64 & \textcolor{ForestGreen}{government} & 33 & 39.0482 \\
352 & \textcolor{ForestGreen}{tax} & 30 & 34.9038 \\
242 & candidate & 18 & 34.3173 \\
266 & \textcolor{ForestGreen}{force} & 28 & 33.8163 \\
66 & propose & 26 & 32.3521 \\
100 & \textcolor{ForestGreen}{federal} & 35 & 31.6903 \\
116 & state & 22 & 31.1115 \\
\hline
\end{tabular}

\vspace{0.2cm}
\textbf{Top 10 por Grado con pesos}
\end{minipage}

\vspace{0.5cm}
El núcleo central es estable, ya que ambas tablas comparten el 70\% de palabras en común 
entre clasificaciones, tanto cuando se mide la diversidad de conexiones como cuando se 
evalúa su intensidad acumulada.

Sin embargo, el cambio más significativo ocurre con el término \textit{voter}, que en la clasificación
de grado con peso se observa en agrupación junto a \textit{government}. Esto indica un posible
discurso electoral vinculado a cuestiones institucionales y fiscales por su 
similitud en pesos con los términos \textit{tax}, \textit{candidate}, 
\textit{force}, \textit{propose}, \textit{federal} y 
\textit{state}.

Términos como \textit{taxpayer} o \textit{method} mantienen vínculos fuertes y frecuentes 
dada su posición en el ranking y correlación con términos de alto grado en peso.
 
Si bien todas las palabras son léxicamente distintas, se pueden formar grupos de palabras 
que definen un proceso electoral y, además, connotan un contexto político posiblemente
enfocado en propuestas electorales. Es por ello que la palabra que más define el conjunto
es \textit{election}, el cual es un término más representativo para esta comunidad y sintetiza
el significado común de los nodos centrales.
\newpage


\section*{Comunidad 1 - Prision}

En la Comunidad 1 se observa una estructura relativamente compacta donde los términos 
\textit{term}, \textit{prison}, \textit{permit} y \textit{approve} ocupan posiciones destacadas en ambas 
métricas, lo que indica que actúan como núcleos estructurales y semánticos dentro de la red. 
Estos nodos presentan simultáneamente una diversidad notable de conexiones y una intensidad 
que, en términos de valores, presenta una diferencia numérica mayor. Esto sugiere que todos 
forman parte del vocabulario central de esta comunidad, otorgándole a \textit{prison}
la mayor carga descriptiva.

\vspace{0.5cm}
\noindent
\begin{minipage}[t]{0.48\textwidth}
\centering
\begin{tabular}{r l r r}
\hline
Id & Label & Grado & \shortstack{Grado \\ con pesos} \\
\hline
20 & \textcolor{ForestGreen}{term} & 21 & 15.9108 \\
206 & \textcolor{ForestGreen}{prison} & 17 & 21.1576 \\
198 & \textcolor{ForestGreen}{permit} & 14 & 13.3879 \\
67 & \textcolor{ForestGreen}{believe} & 12 & 11.9297 \\
316 & \textcolor{ForestGreen}{approve} & 12 & 16.0187 \\
517 & \textcolor{ForestGreen}{violate} & 12 & 10.0479 \\
808 & crisis & 12 & 4.8851 \\
705 & remainder & 11 & 7.1639 \\
265 & warn & 10 & 4.7316 \\
1933 & split & 10 & 6.8805 \\
\hline
\end{tabular}

\vspace{0.2cm}
\textbf{Top 10 por Grado}
\end{minipage}
\hspace{0.5cm}
\begin{minipage}[t]{0.48\textwidth}
\centering
\begin{tabular}{r l r r}
\hline
Id & Label & Grado & \shortstack{Grado \\ con pesos} \\
\hline
206 & \textcolor{ForestGreen}{prison} & 17 & 21.1576 \\
316 & \textcolor{ForestGreen}{approve} & 12 & 16.0187 \\
20 & \textcolor{ForestGreen}{term} & 21 & 15.9108 \\
315 & bond & 8 & 14.1677 \\
198 & \textcolor{ForestGreen}{permit} & 14 & 13.3879 \\
67 & \textcolor{ForestGreen}{believe} & 12 & 11.9297 \\
517 & \textcolor{ForestGreen}{violate} & 12 & 10.0479 \\
789 & courtroom & 6 & 8.8003 \\
865 & vindication & 6 & 8.8003 \\
920 & stagger & 6 & 8.8003 \\
\hline
\end{tabular}

\vspace{0.2cm}
\textbf{Top 10 por Grado con pesos}
\end{minipage}
\vspace{0.5cm}

Seis de los diez términos coinciden en ambas tablas, 
lo que refleja coherencia temática entre los nodos más conectados y relacionados. 
Esta coincidencia indica que los conceptos dominantes mantienen su relevancia 
independientemente de la existencia de otros términos, existiendo una amplitud estructural 
con un fuerte vínculo semántico.
Destaca \textit{bond} (fianza), que no aparece entre los nodos con 
mayor grado pero sí alcanza una posición alta en el ranking ponderado, lo que indica que, aunque 
se relaciona con menos términos distintos, sus conexiones son especialmente intensas dentro del 
contexto de la comunidad. De forma similar, \textit{courtroom}, \textit{vindication} y \textit{stagger}\footnote{
que podría venir de \textit{staggering sentence} (sentencia desproporcionada) o \textit{stagger payments} (pagos prorrateados).}
aparecen solo en el listado ponderado como términos formales, lo que sugiere que refuerza un entorno jurídico
y de ámbito penal.

Por el contrario, términos como \textit{crisis}, \textit{remainder}, \textit{warn} o 
\textit{split} aparecen únicamente en el top por grado, lo que indica que poseen una mayor 
diversidad de conexiones pero con menor intensidad total, actuando como nodos de enlace o 
términos más generales que apuntan a procesos normativos, decisiones formales y situaciones 
de carácter jurídico. Por este 
motivo, el término más representativo para esta comunidad es \textit{prison}, ya que 
el ascenso de la palabra \textit{bond} (fianza) refuerza la interpretación 
legal y punitiva que se le puede atribuir a la temática dominante.



\section*{Comunidad 2 - Investigate}

En la Comunidad 2 la estructura no presenta un núcleo tan marcado como en otras comunidades. 
Aunque \textit{sewer}, \textit{brief}, \textit{adjustment} e \textit{investigate} 
aparecen en ambas métricas, el grado con pesos muestra una distribución más homogénea 
entre varios términos, sin un liderazgo tan acusado, donde en el ranking ponderado 
se introducen términos distintos con mayor intensidad.
\vspace{0.5cm}
\noindent
\begin{minipage}[t]{0.48\textwidth}
\centering
\begin{tabular}{r l r r}
\hline
Id & Label & Grado & \shortstack{Grado \\ con pesos} \\
\hline
1498 & \textcolor{ForestGreen}{sewer} & 19 & 14.7579 \\
208 & file & 17 & 9.0250 \\
454 & gun & 17 & 4.5346 \\
458 & sheriff & 17 & 8.1063 \\
233 & \textcolor{ForestGreen}{brief} & 16 & 12.0738 \\
324 & \textcolor{ForestGreen}{adjustment} & 16 & 12.0738 \\
3661 & corporate & 16 & 10.2493 \\
21 & \textcolor{ForestGreen}{investigate} & 15 & 13.3669 \\
202 & pension & 15 & 10.0948 \\
1731 & object & 15 & 7.5810 \\
\hline
\end{tabular}

\vspace{0.2cm}
\textbf{Top 10 por Grado}
\end{minipage}
\hspace{0.5cm}
\begin{minipage}[t]{0.48\textwidth}
\centering
\begin{tabular}{r l r r}
\hline
Id & Label & Grado & \shortstack{Grado \\ con pesos} \\
\hline
1345 & hire & 11 & 16.3325 \\
1470 & assure & 10 & 15.0955 \\
1498 & \textcolor{ForestGreen}{sewer} & 19 & 14.7579 \\
383 & reject & 13 & 13.4912 \\
378 & privilege & 11 & 13.3967 \\
21 & \textcolor{ForestGreen}{investigate} & 15 & 13.3669 \\
204 & employe & 12 & 12.8397 \\
233 & \textcolor{ForestGreen}{brief} & 16 & 12.0738 \\
324 & \textcolor{ForestGreen}{adjustment} & 16 & 12.0738 \\
1497 & provision & 12 & 11.3156 \\
\hline
\end{tabular}

\vspace{0.2cm}
\textbf{Top 10 por Grado con pesos}
\end{minipage}
\vspace{0.5cm}
El término \textit{sewer} posee el mayor grado y es el tercer término con mayor grado 
con pesos. Esto lo convierte en un nodo estructural fuerte que, si aparece junto a 
palabras como \textit{adjustment}, \textit{corporate}, \textit{pension}, 
\textit{provision} o \textit{investigate}, podría estar vinculado a gestión pública, 
contratos municipales o revisiones administrativas relacionadas con servicios urbanos.
Ees un sustantivo concreto que puede ser candidato a 
término representativo, pero deja sin representar a varios elementos de la comunidad. 

Uno de los términos que más destaca es \textit{gun}, que aparece en grado simple 
pero no en el ranking ponderado, lo que hace probable que se conecte con muchos términos 
distintos pero no forme parte del núcleo temático más cohesionado, situándolo como 
término contextual, no central. Podría estar relacionado con regulación o licencias, 
si en el corpus existen textos administrativos sobre armas.
La comunidad en general combina nodos con alta diversidad de conexiones 
(\textit{file}, \textit{gun}, \textit{sheriff}, \textit{pension}). 
Si analizamos la proximidad numérica en el grado con pesos, se sugiere un entorno 
temático compartido que aborda diferentes partes de un mismo tema. El conjunto apunta a procesos 
administrativos y procedimentales, donde investigación, contratación, revisión y 
decisiones formales forman parte del mismo núcleo.

Por otro lado, \textit{hire} y \textit{assure} son menos estructurales aunque lideren 
el peso ponderado. Por ello, el término más representativo para esta comunidad es 
\textit{investigate}, ya que aparece entre los nodos con mayor grado y peso y podría 
englobar el resto de relaciones ya que si el campo semántico dominante es 
procedimental o administrativo, funciona como verbo articulador.

\newpage



\section*{Comunidad 3 - Police}
En la Comunidad 3 se observa una estructura jerárquica, 
ya que \textit{police} presenta una diferencia numérica amplia 
respecto al resto de términos en el grado con pesos, consolidándose como núcleo dominante.
Otros términos como \textit{construction}, \textit{hearing}, \textit{attorney}, \textit{admit}, 
\textit{cabinet} y \textit{property} ocupan posiciones destacadas en ambas métricas, 
constituyendo el núcleo estructural y semántico del subgrafo. Estos nodos combinan 
una alta diversidad de conexiones con una intensidad elevada de coaparición, situándose como 
centros semánticos de la red.


\vspace{0.5cm}
\noindent
\begin{minipage}[t]{0.48\textwidth}
\centering
\begin{tabular}{r l r r}
\hline
Id & Label & Grado & \shortstack{Grado \\ con pesos} \\
\hline
772 & \textcolor{ForestGreen}{police} & 47 & 59.9574 \\
331 & \textcolor{ForestGreen}{construction} & 34 & 36.8864 \\
470 & \textcolor{ForestGreen}{hearing} & 33 & 30.2936 \\
219 & \textcolor{ForestGreen}{attorney} & 26 & 31.0277 \\
798 & \textcolor{ForestGreen}{admit} & 25 & 29.5414 \\
222 & \textcolor{ForestGreen}{property} & 24 & 25.6138 \\
777 & arrest & 24 & 22.1554 \\
1000 & \textcolor{ForestGreen}{cabinet} & 24 & 26.2667 \\
1 & investigation & 23 & 16.9025 \\
137 & court & 23 & 14.4315 \\
\hline
\end{tabular}

\vspace{0.2cm}
\textbf{Top 10 por Grado}
\end{minipage}
\hspace{0.5cm}
\begin{minipage}[t]{0.48\textwidth}
\centering
\begin{tabular}{r l r r}
\hline
Id & Label & Grado & \shortstack{Grado \\ con pesos} \\
\hline
772 & \textcolor{ForestGreen}{police} & 47 & 59.9574 \\
331 & \textcolor{ForestGreen}{construction} & 34 & 36.8864 \\
10 & jury & 19 & 31.4183 \\
219 & \textcolor{ForestGreen}{attorney} & 26 & 31.0277 \\
470 & \textcolor{ForestGreen}{hearing} & 33 & 30.2936 \\
798 & \textcolor{ForestGreen}{admit} & 25 & 29.5414 \\
138 & fire & 17 & 29.2208 \\
28 & receive & 21 & 28.1654 \\
1000 & \textcolor{ForestGreen}{cabinet} & 24 & 26.2667 \\
222 & \textcolor{ForestGreen}{property} & 24 & 25.6138 \\
\hline
\end{tabular}

\vspace{0.2cm}
\textbf{Top 10 por Grado con pesos}
\end{minipage}

\vspace{0.5cm}


El núcleo principal presenta una coincidencia elevada entre ambas tablas, 
con siete de los diez términos en común en ambos rankings. Esto indica que 
los conceptos centrales mantienen relevancia tanto en la amplitud de 
relaciones como en la intensidad total de sus vínculos.

Entre las diferencias más relevantes se observa que términos como 
\textit{arrest}, \textit{investigation} y \textit{court} aparecen únicamente 
en el listado por grado, lo que indica que se relacionan con un número amplio 
de términos distintos pero con menor intensidad acumulada. Mientras tanto, 
palabras como \textit{jury}, \textit{fire} y \textit{receive} solo aparecen 
en el ranking ponderado, lo que indica relaciones más intensas aunque con 
menos conexiones distintas. El ascenso de \textit{jury} en el ranking 
ponderado refuerza la dimensión judicial del clúster, ya que mantiene 
vínculos intensos con los términos centrales.

El campo semántico integra el ámbito penal y judicial con elementos de 
gestión institucional, lo que sugiere un entorno vinculado a seguridad 
pública y administración formal. Por este motivo, el término más 
representativo para esta comunidad es \textit{police}, ya que aparece 
como el nodo con mayor grado y peso y sintetiza el eje temático dominante.
\newpage


\section*{Comunidad 4 - Diplomatic}
En la Comunidad 4 se observa una estructura con un núcleo dominante moderado, 
donde \textit{consultation} presenta el mayor grado y el mayor peso, situándose 
como punto central del subgrafo. Junto a \textit{ally}, \textit{diplomatic} y 
\textit{prevent}, forma el bloque principal que concentra tanto diversidad de 
conexiones como intensidad de coocurrencia. No obstante, tras 
\textit{consultation} los valores del grado con pesos se agrupan en un rango 
próximo, lo que indica una cohesión compartida más que una jerarquía muy marcada.

\vspace{0.5cm}
\noindent
\begin{minipage}[t]{0.48\textwidth}
\centering
\begin{tabular}{r l r r}
\hline
Id & Label & Grado & \shortstack{Grado \\ con pesos} \\
\hline
1149 & \textcolor{ForestGreen}{consultation} & 49 & 41.8272 \\
1037 & \textcolor{ForestGreen}{ally} & 41 & 34.0270 \\
1229 & \textcolor{ForestGreen}{diplomatic} & 41 & 34.0270 \\
1238 & intervention & 38 & 28.3684 \\
1228 & bloc & 35 & 27.7394 \\
1297 & neutralist & 35 & 27.7394 \\
1057 & bitter & 32 & 22.5169 \\
1051 & tension & 31 & 19.7045 \\
1175 & \textcolor{ForestGreen}{prevent} & 30 & 35.9384 \\
1061 & allied & 29 & 23.5419 \\
\hline
\end{tabular}

\vspace{0.2cm}
\textbf{Top 10 por Grado}
\end{minipage}
\hspace{0.5cm}
\begin{minipage}[t]{0.48\textwidth}
\centering
\begin{tabular}{r l r r}
\hline
Id & Label & Grado & \shortstack{Grado \\ con pesos} \\
\hline
1149 & \textcolor{ForestGreen}{consultation} & 49 & 41.8272 \\
1175 & \textcolor{ForestGreen}{prevent} & 30 & 35.9384 \\
972 & reaction & 26 & 34.2553 \\
1037 & \textcolor{ForestGreen}{ally} & 41 & 34.0270 \\
1229 & \textcolor{ForestGreen}{diplomatic} & 41 & 34.0270 \\
169 & political & 27 & 33.9719 \\
1022 & climate & 24 & 33.1499 \\
1088 & global & 24 & 32.5985 \\
1148 & timely & 24 & 32.5985 \\
1184 & inclination & 24 & 32.5985 \\
\hline
\end{tabular}

\vspace{0.2cm}
\textbf{Top 10 por Grado con pesos}
\end{minipage}
\vspace{0.5cm}
 
Entre las variaciones más relevantes se observa que términos como 
\textit{intervention}, \textit{bloc}, \textit{neutralist}, \textit{bitter}, 
\textit{tension} y \textit{allied} aparecen únicamente en el ranking por grado. 
Esto indica que se relacionan con un número amplio de términos distintos, aunque 
con menor intensidad acumulada, actuando como elementos de contexto general.

Por el contrario, palabras como \textit{reaction}, \textit{political}, 
\textit{climate}, \textit{global}, \textit{timely} o \textit{inclination} solo 
aparecen en el listado ponderado. Esto sugiere que mantienen relaciones más 
intensas dentro de contextos específicos y refuerzan una dimensión más 
abstracta y estratégica del discurso diplomático. Destaca especialmente 
\textit{prevent}, cuyo peso es significativamente alto en relación con su grado, 
lo que indica una función relevante en la cohesión temática.

El conjunto apunta a relaciones internacionales, alianzas y mecanismos de 
cooperación orientados a evitar conflictos. Aunque \textit{consultation} es el 
término estructuralmente más fuerte, \textit{diplomatic} sintetiza mejor el 
ámbito temático general de la comunidad. Por este motivo, el término más 
representativo para este clúster es \textit{diplomatic}.

 

\section*{Comunidad 5 - Estimate}
En la Comunidad 5 se observa una estructura relativamente equilibrada. 
\textit{estimate} ocupa la primera posición en ambas métricas, pero la 
diferencia con \textit{proposal} en el grado con pesos es mínima, lo que 
indica un núcleo compartido más que un liderazgo aislado. 
Términos como \textit{residential}, \textit{outlay}, \textit{approval}, 
\textit{victim} e \textit{increase} conforman un bloque fuerte al presentar 
valores ponderados próximos entre sí.

\vspace{0.5cm}
\noindent
\begin{minipage}[t]{0.48\textwidth}
\centering
\begin{tabular}{r l r r}
\hline
Id & Label & Grado & \shortstack{Grado \\ con pesos} \\
\hline
482 & \textcolor{ForestGreen}{estimate} & 29 & 24.4942 \\
559 & \textcolor{ForestGreen}{outlay} & 21 & 18.3271 \\
554 & \textcolor{ForestGreen}{residential} & 20 & 19.7080 \\
2062 & flat & 20 & 13.9847 \\
1192 & chain & 19 & 13.6469 \\
2090 & builder & 19 & 7.0864 \\
1067 & atomic & 18 & 13.7516 \\
1426 & shopping & 17 & 8.8281 \\
2076 & \textcolor{ForestGreen}{victim} & 17 & 17.3913 \\
459 & \textcolor{ForestGreen}{approval} & 16 & 17.4590 \\
\hline
\end{tabular}

\vspace{0.2cm}
\textbf{Top 10 por Grado}
\end{minipage}
\hspace{0.5cm}
\begin{minipage}[t]{0.48\textwidth}
\centering
\begin{tabular}{r l r r}
\hline
Id & Label & Grado & \shortstack{Grado \\ con pesos} \\
\hline
482 & \textcolor{ForestGreen}{estimate} & 29 & 24.4942 \\
581 & proposal & 14 & 24.3684 \\
554 & \textcolor{ForestGreen}{residential} & 20 & 19.7080 \\
559 & \textcolor{ForestGreen}{outlay} & 21 & 18.3271 \\
459 & \textcolor{ForestGreen}{approval} & 16 & 17.4590 \\
2076 & \textcolor{ForestGreen}{victim} & 17 & 17.3913 \\
364 & increase & 14 & 17.2862 \\
498 & personal & 10 & 14.3840 \\
904 & payment & 13 & 14.1531 \\
610 & requirement & 12 & 14.0937 \\
\hline
\end{tabular}

\vspace{0.2cm}
\textbf{Top 10 por Grado con pesos}
\end{minipage}
\vspace{0.5cm}
 

Entre las diferencias más relevantes se observa que términos como 
\textit{flat}, \textit{chain}, \textit{builder}, \textit{atomic} y 
\textit{shopping} aparecen únicamente en el listado por grado, lo que 
indica conexiones amplias pero menos intensas.

Por el contrario, palabras como \textit{proposal}, \textit{increase}, 
\textit{personal}, \textit{payment} o \textit{requirement} solo aparecen 
en el ranking ponderado, lo que señala relaciones más concentradas y 
recurrentes en contextos específicos.

El término \textit{victim} es relevante porque mantiene grado y peso similares a 
\textit{approval} e \textit{increase}, lo que sugiere un proceso 
administrativo donde se estiman daños, se formulan propuestas y se 
aprueban compensaciones, posiblemente en entornos residenciales.

El campo semántico se vincula a procedimientos económicos y formales, por lo que 
el término más representativo es \textit{estimate}, ya que lidera en 
grado y peso y articula el conjunto temático del subgrafo.

\newpage

\section*{Comunidad 6 - Mechanism}

En la Comunidad 6 se observa una estructura relativamente cohesionada donde los términos 
\textit{stem}, \textit{very}, \textit{vary} y \textit{mechanism} aparecen en posiciones destacadas 
en ambas métricas como núcleo estructural y semántico del subgrafo. 
La palabra \textit{very} tiende a aparecer de forma extremadamente recurrente, lo que posiblemente 
sea debido a apariciones junto a términos técnicos de esta comunidad, como artificial, 
protective o mechanism, para reforzar descripciones por lo que su co-aparición no es 
aleatoria, sino que está anclada a conceptos técnicos de este clúster específico; su 
naturaleza adverbial sugiere un uso intensificador más que conceptual.

\vspace{0.5cm}
\noindent
\begin{minipage}[t]{0.48\textwidth}
\centering
\begin{tabular}{r l r r}
\hline
Id & Label & Grado & \shortstack{Grado \\ con pesos} \\
\hline
1332 & \textcolor{ForestGreen}{stem} & 23 & 14.7057 \\
1016 & \textcolor{ForestGreen}{very} & 22 & 20.5590 \\
3389 & \textcolor{ForestGreen}{vary} & 20 & 16.1942 \\
3947 & \textcolor{ForestGreen}{mechanism} & 20 & 13.7127 \\
1431 & enterprise & 18 & 10.6537 \\
3768 & profession & 18 & 10.5577 \\
1126 & european & 17 & 9.0358 \\
1039 & circumstance & 15 & 7.6644 \\
1111 & application & 15 & 10.1116 \\
168 & eliminate & 14 & 6.1563 \\
\hline
\end{tabular}

\vspace{0.2cm}
\textbf{Top 10 por Grado}
\end{minipage}
\hspace{0.5cm}
\begin{minipage}[t]{0.48\textwidth}
\centering
\begin{tabular}{r l r r}
\hline
Id & Label & Grado & \shortstack{Grado \\ con pesos} \\
\hline
1016 & \textcolor{ForestGreen}{very} & 22 & 20.5590 \\
4968 & scheme & 11 & 16.7398 \\
4987 & protective & 11 & 16.7398 \\
5014 & brand & 11 & 16.7398 \\
5029 & paradox & 11 & 16.7398 \\
5045 & substitute & 11 & 16.7398 \\
5059 & artificial & 11 & 16.7398 \\
3389 & \textcolor{ForestGreen}{vary} & 20 & 16.1942 \\
1332 & \textcolor{ForestGreen}{stem} & 23 & 14.7057 \\
3947 & \textcolor{ForestGreen}{mechanism} & 20 & 13.7127 \\
\hline
\end{tabular}

\vspace{0.2cm}
\textbf{Top 10 por Grado con pesos}
\end{minipage}
\vspace{0.5cm}


El ranking ponderado muestra un bloque compacto formado por 
\textit{scheme}, \textit{protective}, \textit{brand}, \textit{paradox}, 
\textit{substitute} y \textit{artificial}, todos con los mismos valores, 
lo que indica cohesión temática en torno a conceptos técnicos o abstractos, donde 
\textit{vary}, \textit{stem} y \textit{mechanism} se integran en este 
entorno conceptual relacionado con procesos, variación y funcionamiento 
de sistemas.

El campo semántico sugiere un ámbito conceptual asociado a mecanismos, cambios y condiciones de operación. Por 
este motivo, el término más representativo es \textit{mechanism}, ya que aparece entre los nodos con 
mayor grado y peso y sintetiza el eje temático dominante del subgrafo.

\newpage


\section*{Comunidad 7 - Taxiing}

En la Comunidad 7 se observa una estructura parcialmente concentrada donde los términos 
\textit{automatic}, \textit{taxi} y \textit{halt} sugieren una comunidad con 
conexiones variadas en el grado simple.
Esto refleja una amplitud estructural alta pero sin que el núcleo temático 
quede claramente definido únicamente por esta métrica.
Sin embargo, al aplicar la métrica de grado con pesos, la estructura muestra 
una característica significativa; aparece un bloque sólido de términos relacionados con la aeronáutica: 
\textit{cockpit}, \textit{bomber}, \textit{runway}, \textit{jet}, 
\textit{pilot}, \textit{engine} y \textit{landing}, que poseen exactamente el mismo peso 
de interacción. Esta igualdad numérica indica una coaparición recurrente y concentrada, 
propia de un subgrupo altamente cohesionado dentro del subgrafo. Estos términos solo 
aparecen en el ranking ponderado, lo que refuerza que mantienen relaciones más intensas 
aunque con menos conexiones distintas, rasgo característico de términos especializados.

\vspace{0.5cm}
\noindent
\begin{minipage}[t]{0.48\textwidth}
\centering
\begin{tabular}{r l r r}
\hline
Id & Label & Grado & \shortstack{Grado \\ con pesos} \\
\hline
3321 & \textcolor{ForestGreen}{automatic} & 24 & 19.0645 \\
3881 & farmer & 22 & 14.5975 \\
1721 & \textcolor{ForestGreen}{taxi} & 21 & 22.6571 \\
3567 & fuel & 21 & 16.9585 \\
1253 & disclose & 19 & 12.8941 \\
1326 & crime & 17 & 14.2169 \\
1762 & mortgage & 15 & 11.7213 \\
817 & write & 13 & 6.1059 \\
1449 & wage & 13 & 12.6424 \\
3921 & \textcolor{ForestGreen}{halt} & 12 & 19.3924 \\
\hline
\end{tabular}

\vspace{0.2cm}
\textbf{Top 10 por Grado}
\end{minipage}
\hspace{0.5cm}
\begin{minipage}[t]{0.48\textwidth}
\centering
\begin{tabular}{r l r r}
\hline
Id & Label & Grado & \shortstack{Grado \\ con pesos} \\
\hline
1721 & \textcolor{ForestGreen}{taxi} & 21 & 22.6571 \\
3921 & \textcolor{ForestGreen}{halt} & 12 & 19.3924 \\
3928 & cockpit & 12 & 19.3924 \\
3932 & bomber & 12 & 19.3924 \\
3933 & runway & 12 & 19.3924 \\
3936 & jet & 12 & 19.3924 \\
3939 & pilot & 12 & 19.3924 \\
3950 & engine & 12 & 19.3924 \\
3966 & landing & 12 & 19.3924 \\
3321 & \textcolor{ForestGreen}{automatic} & 24 & 19.0645 \\
\hline
\end{tabular}

\vspace{0.2cm}
\textbf{Top 10 por Grado con pesos}
\end{minipage}
\vspace{0.5cm}
 

El término más llamativo es \textit{taxi}, que no debe interpretarse como 
``taxi urbano'', sino como \textit{taxiing}, la maniobra de desplazamiento 
de una aeronave en pista. La presencia de \textit{bomber} y \textit{jet} 
refuerza que no se trata de vehículos terrestres, sino de aeronaves específicas. 
\textit{Taxi} aparece en ambas métricas y lidera el grado con pesos, lo que lo 
convierte en el nodo de mayor intensidad dentro del bloque aeronáutico.

El término \textit{halt}, que ocupa una posición baja en grado simple, escala hasta la 
segunda posición ponderada, lo que demuestra que su uso está estrechamente vinculado a 
este contexto específico de la aviación. Los términos como \textit{farmer}, 
\textit{fuel}, \textit{disclose}, \textit{crime}, \textit{mortgage}, \textit{write} y 
\textit{wage} que solo figuran en el grado 
simple mantienen amplitud estructural, pero el peso los excluye del núcleo temático 
cohesionado.

El campo semántico apunta a un ámbito relacionado con la aviación, 
especialmente asociado a maniobras y operaciones de vuelo. 
Por este motivo, el término más representativo es \textit{Taxiing}, ya que 
lidera en intensidad y articula el bloque técnico dominante del subgrafo.

\newpage

\section*{Comunidad 8 - Development}

En la Comunidad 8 se observa una estructura cohesionada donde los términos 
\textit{advance}, \textit{understand}, \textit{development}, \textit{understanding}, 
\textit{reference}, \textit{process}, \textit{weakness} y \textit{budget} aparecen en posiciones 
destacadas en ambas métricas, lo que indica que constituyen el núcleo estructural y semántico del 
subgrafo. Estos nodos combinan una diversidad notable de conexiones con una intensidad elevada de 
coaparición, situándose como los principales organizadores conceptuales dentro de la red.

\vspace{0.5cm}
\noindent
\begin{minipage}[t]{0.48\textwidth}
\centering
\begin{tabular}{r l r r}
\hline
Id & Label & Grado & \shortstack{Grado \\ con pesos} \\
\hline
1014 & \textcolor{ForestGreen}{understanding} & 24 & 15.5003 \\
1060 & \textcolor{ForestGreen}{advance} & 21 & 32.8997 \\
1026 & \textcolor{ForestGreen}{understand} & 20 & 28.7065 \\
4537 & adequate & 19 & 10.4317 \\
670 & \textcolor{ForestGreen}{development} & 17 & 25.1008 \\
462 & certain & 15 & 9.1241 \\
1567 & \textcolor{ForestGreen}{reference} & 14 & 12.7389 \\
862 & \textcolor{ForestGreen}{process} & 13 & 14.4028 \\
1697 & \textcolor{ForestGreen}{weakness} & 13 & 16.2344 \\
555 & \textcolor{ForestGreen}{budget} & 12 & 14.9643 \\
\hline
\end{tabular}

\vspace{0.2cm}
\textbf{Top 10 por Grado}
\end{minipage}
\hspace{0.5cm}
\begin{minipage}[t]{0.48\textwidth}
\centering
\begin{tabular}{r l r r}
\hline
Id & Label & Grado & \shortstack{Grado \\ con pesos} \\
\hline
1060 & \textcolor{ForestGreen}{advance} & 21 & 32.8997 \\
1026 & \textcolor{ForestGreen}{understand} & 20 & 28.7065 \\
670 & \textcolor{ForestGreen}{development} & 17 & 25.1008 \\
1697 & \textcolor{ForestGreen}{weakness} & 13 & 16.2344 \\
1014 & \textcolor{ForestGreen}{understanding} & 24 & 15.5003 \\
555 & \textcolor{ForestGreen}{budget} & 12 & 14.9643 \\
862 & \textcolor{ForestGreen}{process} & 13 & 14.4028 \\
1567 & \textcolor{ForestGreen}{reference} & 14 & 12.7389 \\
1142 & apparent & 10 & 11.4902 \\
1035 & difficulty & 10 & 11.3314 \\
\hline
\end{tabular}

\vspace{0.2cm}
\textbf{Top 10 por Grado con pesos}
\end{minipage}
\vspace{0.5cm}

Entre las diferencias más relevantes se observa que términos como \textit{adequate} y 
\textit{certain} aparecen únicamente en el listado por grado, lo que indica que se relacionan con 
muchos términos distintos pero con menor intensidad total que palabras como \textit{apparent} y 
\textit{difficulty} que solo aparecen en el ranking ponderado. Esto sugiere una 
especialización temática hacia el análisis y la resolución de problemas, debido a que 
estas últimas mantienen relaciones más intensas aunque con menos conexiones distintas, 
rasgo característico de términos más específicos o dependientes de contextos concretos.

El campo semántico sugiere un ámbito conceptual vinculado a procesos cognitivos, desarrollo y análisis, 
donde aparecen términos relacionados con comprensión, evaluación y evolución conceptual. Por este 
motivo, el término más representativo es \textit{development}, ya que 
permite identificar los términos que sostienen la carga semántica real del grupo, 
centrada en el avance, el desarrollo procedimental, además que aparece entre los nodos con mayor 
grado y peso, sintetizando el eje temático dominante del subgrafo.


\newpage
\section*{Comunidad 9 - Pledge}

En la Comunidad 9 se observa una estructura moderadamente dispersa donde los términos 
\textit{pledge}, \textit{tract}, \textit{animal} y \textit{teamster} aparecen en posiciones 
destacadas en ambas métricas, lo que indica que constituyen el núcleo estructural y semántico del 
subgrafo. Estos nodos combinan una diversidad de conexiones relativamente alta con una intensidad 
de coaparición significativa, situándose como los principales organizadores dentro de la red.

\vspace{0.5cm}
\noindent
\begin{minipage}[t]{0.48\textwidth}
\centering
\begin{tabular}{r l r r}
\hline
Id & Label & Grado & \shortstack{Grado \\ con pesos} \\
\hline
1636 & \textcolor{ForestGreen}{pledge} & 15 & 11.8739 \\
1661 & \textcolor{ForestGreen}{tract} & 10 & 11.7035 \\
2128 & explosive & 10 & 3.9676 \\
425 & calm & 9 & 4.9719 \\
1490 & meaning & 9 & 4.3317 \\
2064 & \textcolor{ForestGreen}{animal} & 9 & 7.2759 \\
2108 & \textcolor{ForestGreen}{teamster} & 9 & 7.2759 \\
441 & violence & 8 & 3.1471 \\
2038 & intervene & 8 & 4.8900 \\
4624 & wash & 8 & 5.2467 \\
\hline
\end{tabular}

\vspace{0.2cm}
\textbf{Top 10 por Grado}
\end{minipage}
\hspace{0.5cm}
\begin{minipage}[t]{0.48\textwidth}
\centering
\begin{tabular}{r l r r}
\hline
Id & Label & Grado & \shortstack{Grado \\ con pesos} \\
\hline
1636 & \textcolor{ForestGreen}{pledge} & 15 & 11.8739 \\
1661 & \textcolor{ForestGreen}{tract} & 10 & 11.7035 \\
1568 & discredited & 6 & 10.5921 \\
1591 & desperate & 6 & 10.5921 \\
1607 & repay & 6 & 10.5921 \\
1657 & hoodlum & 6 & 10.5921 \\
1665 & conservation & 6 & 10.5921 \\
2064 & \textcolor{ForestGreen}{animal} & 9 & 7.2759 \\
2108 & \textcolor{ForestGreen}{teamster} & 9 & 7.2759 \\
17 & manner & 3 & 5.3877 \\
\hline
\end{tabular}

\vspace{0.2cm}
\textbf{Top 10 por Grado con pesos}
\end{minipage}
\vspace{0.5cm}

Entre las diferencias más relevantes se observa que términos como \textit{explosive}, 
\textit{calm}, \textit{meaning}, \textit{violence}, \textit{intervene} y \textit{wash} aparecen 
únicamente en el listado por grado, lo que indica que se relacionan con muchos términos distintos 
pero con menor intensidad total, actuando como nodos contextuales y palabras como \textit{discredited}, \textit{desperate}, \textit{repay}, 
\textit{hoodlum} y \textit{conservation} emergen con un peso idéntico de 10.5921. 
Según la teoría de partición de grafos, este fenómeno indica la existencia de un subgrupo con 
una relación ``más fuerte'' e intensa entre sí que con el resto de la red dado a que solo aparecen en el ranking ponderado, 
lo que indica que mantienen relaciones más intensas aunque con menos conexiones distintas, rasgo 
propio de términos más específicos o dependientes de contextos concretos.
El término más representativo es \textit{pledge} (promesa o compromiso) actúa como un eje 
central para un grupo de palabras con tintes sociales y procedimentales. La aparición 
de términos con alto peso como repay (reembolsar), conservation (conservación) y 
teamster (miembro de sindicato de transporte) sugiere un contexto de acuerdos o 
compromisos institucionales y laborales. Al ser el nodo con mayor centralidad en la matriz de 
adyacencia de este grupo, representa fielmente a una comunidad donde el compromiso o la 
obligación parece ser el vínculo temático predominante.




\newpage 


\section*{Comunidad 10 - Market}

En la Comunidad 10 se observa una estructura claramente definida donde los términos 
\textit{market}, \textit{earning}, \textit{investor}, \textit{production}, \textit{equipment} y 
\textit{price} aparecen en posiciones destacadas en ambas métricas, lo que indica que constituyen 
el núcleo estructural y semántico del subgrafo. Estos nodos combinan una alta diversidad de 
conexiones con una intensidad elevada de coaparición, situándose como los principales ejes 
organizadores dentro de la red.

\vspace{0.5cm}
\noindent
\begin{minipage}[t]{0.48\textwidth}
\centering
\begin{tabular}{r l r r}
\hline
Id & Label & Grado & \shortstack{Grado \\ con pesos} \\
\hline
3384 & \textcolor{ForestGreen}{market} & 35 & 37.6279 \\
4046 & \textcolor{ForestGreen}{earning} & 33 & 20.4574 \\
4032 & \textcolor{ForestGreen}{investor} & 32 & 19.7702 \\
4097 & marketing & 30 & 16.7314 \\
2173 & compare & 28 & 18.6556 \\
4054 & corn & 27 & 13.6037 \\
4094 & \textcolor{ForestGreen}{production} & 27 & 25.6798 \\
1357 & \textcolor{ForestGreen}{equipment} & 25 & 22.4271 \\
4052 & competitive & 24 & 16.0284 \\
179 & \textcolor{ForestGreen}{price} & 23 & 20.4423 \\
\hline
\end{tabular}

\vspace{0.2cm}
\textbf{Top 10 por Grado}
\end{minipage}
\hspace{0.5cm}
\begin{minipage}[t]{0.48\textwidth}
\centering
\begin{tabular}{r l r r}
\hline
Id & Label & Grado & \shortstack{Grado \\ con pesos} \\
\hline
3384 & \textcolor{ForestGreen}{market} & 35 & 37.6279 \\
1448 & product & 17 & 29.9264 \\
4094 & \textcolor{ForestGreen}{production} & 27 & 25.6798 \\
1357 & \textcolor{ForestGreen}{equipment} & 25 & 22.4271 \\
4046 & \textcolor{ForestGreen}{earning} & 33 & 20.4574 \\
179 & \textcolor{ForestGreen}{price} & 23 & 20.4423 \\
1551 & industry & 22 & 20.2054 \\
4032 & \textcolor{ForestGreen}{investor} & 32 & 19.7702 \\
2289 & yield & 12 & 19.2198 \\
3345 & improvement & 23 & 19.0698 \\
\hline
\end{tabular}

\vspace{0.2cm}
\textbf{Top 10 por Grado con pesos}
\end{minipage}
\vspace{0.5cm}

Entre las diferencias más relevantes se observa que términos como \textit{marketing}, 
\textit{compare}, \textit{corn} y \textit{competitive} aparecen únicamente en el listado por grado, 
lo que indica que se relacionan con muchos términos distintos pero con menor intensidad total, 
actuando como nodos de enlace o elementos contextuales amplios dentro de la comunidad.

Por el contrario, palabras como \textit{product}, \textit{industry}, \textit{yield} e 
\textit{improvement} solo aparecen en el ranking ponderado, lo que indica que mantienen relaciones 
más intensas aunque con menos conexiones distintas, rasgo característico de términos más específicos 
o dependientes de contextos concretos.

Los términos como \textit{product}, \textit{production}, \textit{earning}, \textit{price} e \textit{investor} 
definen un escenario comercial, un ámbito claramente vinculado a la actividad productiva y económica, por lo que el
término más representativo para esta comunidad es \textit{market}, ya que al aparecer como el nodo con mayor grado y peso,  
destaca como el principal centro de la red, demostrando que en la comunidad analizada, es el término que 
articula la mayoría de las conversaciones sobre comercio e industria


\newpage
\section*{Comunidad 11 - Front}

En la Comunidad 11 se observa una estructura moderadamente dispersa donde los términos 
\textit{front}, \textit{deadlock}, \textit{rebel}, \textit{heat}, \textit{err} y 
\textit{treatment} aparecen en posiciones destacadas en ambas métricas. Combinan una diversidad 
de conexiones relativamente alta con una intensidad de coaparición significativa, situándose 
como los principales nucleos estructurales la red.
El nodo \textit{deadlock} posee el mayor grado simple, lo que sugiere un término de enlace con 
una mayor variedad de palabras distintas. Sin embargo, al aplicar el grado con pesos, 
el término \textit{front} pasa a tener la relación más fuerte y recurrente dentro del clúster.

\vspace{0.5cm}
\noindent
\begin{minipage}[t]{0.48\textwidth}
\centering
\begin{tabular}{r l r r}
\hline
Id & Label & Grado & \shortstack{Grado \\ con pesos} \\
\hline
2292 & \textcolor{ForestGreen}{deadlock} & 17 & 9.1953 \\
2125 & \textcolor{ForestGreen}{front} & 15 & 9.3025 \\
2170 & \textcolor{ForestGreen}{rebel} & 13 & 8.5288 \\
1451 & catch & 12 & 5.1904 \\
2354 & slug & 12 & 7.8427 \\
2442 & physician & 12 & 5.6775 \\
1824 & tactic & 10 & 4.9889 \\
3169 & \textcolor{ForestGreen}{heat} & 10 & 8.6750 \\
1306 & \textcolor{ForestGreen}{err} & 9 & 8.4916 \\
2510 & \textcolor{ForestGreen}{treatment} & 9 & 8.2077 \\
\hline
\end{tabular}

\vspace{0.2cm}
\textbf{Top 10 por Grado}
\end{minipage}
\hspace{0.5cm}
\begin{minipage}[t]{0.48\textwidth}
\centering
\begin{tabular}{r l r r}
\hline
Id & Label & Grado & \shortstack{Grado \\ con pesos} \\
\hline
2125 & \textcolor{ForestGreen}{front} & 15 & 9.3025 \\
5182 & chinese & 7 & 9.2556 \\
5188 & transport & 7 & 9.2556 \\
5256 & ante & 7 & 9.2556 \\
5266 & presidency & 7 & 9.2556 \\
2292 & \textcolor{ForestGreen}{deadlock} & 17 & 9.1953 \\
3169 & \textcolor{ForestGreen}{heat} & 10 & 8.6750 \\
2170 & \textcolor{ForestGreen}{rebel} & 13 & 8.5288 \\
1306 & \textcolor{ForestGreen}{err} & 9 & 8.4916 \\
2510 & \textcolor{ForestGreen}{treatment} & 9 & 8.2077 \\
\hline
\end{tabular}

\vspace{0.2cm}
\textbf{Top 10 por Grado con pesos}
\end{minipage}
\vspace{0.5cm}

Entre las diferencias más relevantes se observa que términos como \textit{catch}, \textit{slug}, 
\textit{physician} y \textit{tactic} aparecen únicamente en el listado por grado, lo que indica 
que se relacionan con muchos términos distintos pero con menor intensidad total, actuando como 
nodos contextuales amplios dentro de la comunidad.

Por el contrario, palabras como \textit{chinese}, \textit{transport}, \textit{ante} y 
\textit{presidency} solo aparecen en el ranking ponderado con un valor idéntico y muy elevado.
lo que sugiere la existencia de una relación entre ellos muy densa donde estos términos 
co-aparecen siempre juntos, formando un bloque temático específico que el grado simple no alcanza a priorizar.

El campo semántico sugiere un conjunto heterogéneo vinculado a situaciones de confrontación, 
posición o dinámica estratégica, donde aparecen términos asociados a frentes, conflictos, 
decisiones y reacciones. Por este motivo, el término más representativo es \textit{front}, 
que destaca por su estabilidad en ambas métricas, ocupando la segunda posición en grado 
simple y ascendiendo al primer puesto en grado con pesos, lo que lo consolida como el núcleo de mayor intensidad de la comunidad.
\newpage


\section*{Comunidad 12 - Catcher}

La Comunidad 12 presenta un núcleo temático muy específico relacionado con el béisbol. 
La aparición de términos con alto peso como \textit{bunt} (toque de bola), 
\textit{shortstop} (paracorto) y \textit{teammate} (compañero de equipo), 
junto con nombres propios que actúan como términos en este corpus, refuerza 
esta interpretación. 

Esto es relevante porque en inglés \textit{mantle}, probablemente en referencia 
a Mickey Mantle, significa \textit{manto}, cuyo significado no guarda relación 
con el resto de palabras salvo que se trate del apellido de un jugador. Esto 
da más fuerza a la definición de un clúster deportivo.

\vspace{0.5cm}
\noindent
\begin{minipage}[t]{0.48\textwidth}
\centering
\begin{tabular}{r l r r}
\hline
Id & Label & Grado & \shortstack{Grado \\ con pesos} \\
\hline
2851 & drag & 18 & 9.2876 \\
2891 & \textcolor{ForestGreen}{blond} & 17 & 15.4827 \\
2894 & \textcolor{ForestGreen}{ultimate} & 17 & 12.2663 \\
2901 & \textcolor{ForestGreen}{possess} & 17 & 15.4827 \\
2917 & \textcolor{ForestGreen}{mantle} & 17 & 15.4827 \\
2360 & pound & 16 & 9.7960 \\
2897 & peak & 16 & 9.8173 \\
2066 & \textcolor{ForestGreen}{catcher} & 15 & 11.3041 \\
2872 & \textcolor{ForestGreen}{recall} & 15 & 9.9340 \\
2324 & \textcolor{ForestGreen}{bunt} & 14 & 12.7305 \\
\hline
\end{tabular}

\vspace{0.2cm}
\textbf{Top 10 por Grado}
\end{minipage}
\hspace{0.5cm}
\begin{minipage}[t]{0.48\textwidth}
\centering
\begin{tabular}{r l r r}
\hline
Id & Label & Grado & \shortstack{Grado \\ con pesos} \\
\hline
2891 & \textcolor{ForestGreen}{blond} & 17 & 15.4827 \\
2901 & \textcolor{ForestGreen}{possess} & 17 & 15.4827 \\
2917 & \textcolor{ForestGreen}{mantle} & 17 & 15.4827 \\
2324 & \textcolor{ForestGreen}{bunt} & 14 & 12.7305 \\
2358 & shortstop & 14 & 12.7305 \\
2390 & teammate & 14 & 12.7305 \\
2894 & \textcolor{ForestGreen}{ultimate} & 17 & 12.2663 \\
2066 & \textcolor{ForestGreen}{catcher} & 15 & 11.3041 \\
2919 & thin & 13 & 10.9846 \\
2872 & \textcolor{ForestGreen}{recall} & 15 & 9.9340 \\
\hline
\end{tabular}

\vspace{0.2cm}
\textbf{Top 10 por Grado con pesos}
\end{minipage}
\vspace{0.5cm}

Entre las diferencias más relevantes se observa que términos como 
\textit{drag}, \textit{pound} y \textit{peak} aparecen únicamente en el 
listado por grado, lo que indica que se relacionan con muchos términos 
distintos pero con menor intensidad total, actuando como nodos 
contextuales amplios dentro de la comunidad.

Por el contrario, palabras como \textit{shortstop}, \textit{teammate} y 
\textit{thin} solo aparecen en el ranking ponderado, lo que indica que 
mantienen relaciones más intensas aunque con menos conexiones distintas, 
rasgo propio de términos más específicos o dependientes de contextos 
concretos.

El campo semántico apunta a un ámbito deportivo, en concreto al béisbol, 
donde aparecen términos asociados a posiciones, jugadas y jugadores. 
Por este motivo, el término más representativo es \textit{Catcher}, ya que 
aparece entre los nodos con mayor grado y peso y sintetiza el eje temático 
dominante del subgrafo. Se descarta \textit{Mantle} porque, aunque podría 
remitir a un jugador concreto, el bloque temático refleja una semántica 
más general del deporte.


\newpage
\section*{Comunidad 13 - League}

En la Comunidad 13 se observa una estructura altamente cohesionada donde los términos 
\textit{player}, \textit{pennant}, \textit{rookie}, \textit{pitcher}, \textit{bat}, 
\textit{ball}, \textit{team}, \textit{league} y \textit{game} aparecen en posiciones destacadas 
en ambas métricas, lo que indica que constituyen el núcleo estructural y semántico del subgrafo. 
Estos nodos combinan una elevada diversidad de conexiones con una intensidad de coaparición muy 
alta, situándose como los principales organizadores dentro de la red.

\vspace{0.5cm}
\noindent
\begin{minipage}[t]{0.48\textwidth}
\centering
\begin{tabular}{r l r r}
\hline
Id & Label & Grado & \shortstack{Grado \\ con pesos} \\
\hline
2357 & \textcolor{ForestGreen}{player} & 63 & 141.2842 \\
2286 & \textcolor{ForestGreen}{pennant} & 54 & 91.2147 \\
2303 & \textcolor{ForestGreen}{rookie} & 54 & 91.2147 \\
2315 & \textcolor{ForestGreen}{pitcher} & 54 & 91.2147 \\
2387 & \textcolor{ForestGreen}{bat} & 54 & 91.2147 \\
2323 & \textcolor{ForestGreen}{ball} & 52 & 118.8597 \\
721 & \textcolor{ForestGreen}{team} & 49 & 80.0137 \\
1488 & \textcolor{ForestGreen}{league} & 49 & 96.7207 \\
2199 & \textcolor{ForestGreen}{game} & 49 & 90.5975 \\
2295 & ninth & 49 & 66.9841 \\
\hline
\end{tabular}

\vspace{0.2cm}
\textbf{Top 10 por Grado}
\end{minipage}
\hspace{0.5cm}
\begin{minipage}[t]{0.48\textwidth}
\centering
\begin{tabular}{r l r r}
\hline
Id & Label & Grado & \shortstack{Grado \\ con pesos} \\
\hline
2357 & \textcolor{ForestGreen}{player} & 63 & 141.2842 \\
2323 & \textcolor{ForestGreen}{ball} & 52 & 118.8597 \\
2307 & pitch & 47 & 112.1459 \\
1488 & \textcolor{ForestGreen}{league} & 49 & 96.7207 \\
2286 & \textcolor{ForestGreen}{pennant} & 54 & 91.2147 \\
2303 & \textcolor{ForestGreen}{rookie} & 54 & 91.2147 \\
2315 & \textcolor{ForestGreen}{pitcher} & 54 & 91.2147 \\
2387 & \textcolor{ForestGreen}{bat} & 54 & 91.2147 \\
2199 & \textcolor{ForestGreen}{game} & 49 & 90.5975 \\
721 & \textcolor{ForestGreen}{team} & 49 & 80.0137 \\
\hline
\end{tabular}

\vspace{0.2cm}
\textbf{Top 10 por Grado con pesos}
\end{minipage}
\vspace{0.5cm}

Entre las diferencias más relevantes se observa que el término \textit{ninth} aparece únicamente 
en el listado por grado y \textit{pitch} solo aparece en el ranking ponderado. 
Esto demuestra que, aunque pitch tiene una menor diversidad de contactos distintos, 
su intensidad de uso junto a los términos centrales como player o ball es significativamente superior
Esta comunidad es un ejemplo de estructura semántica intrínseca, mientras que el grado simple 
refleja la amplitud de la red deportiva, el grado con pesos confirma la existencia de un núcleo de 
términos técnicos y operativos (pitcher, bat, pennant) con una afinidad semántica tan 
alta que sus vínculos se sitúan entre los más pesados de la red completa.

El nodo \textit{player} lidera ambas clasificaciones con una diferencia abrumadora, lo que 
indica que en la matriz de adyacencia de este término posee el mayor número de 
enlaces incidentes y que sus relaciones de co-aparición son las más fuertes y frecuentes de todo el corpus analizado.

El campo semántico es claramente homogéneo y está vinculado al ámbito deportivo, concretamente al 
béisbol, donde aparecen conceptos asociados a jugadores, posiciones, acciones y competición. Por 
este motivo, el término más representativo es \textit{league}, ya que engloba el concepto global
y aparece como uno de los nodos en el ranking  que sintetiza el eje temático dominante del subgrafo.


\newpage

\section*{Comunidad 14 - Academic}

En la Comunidad 14 se observa una estructura moderadamente cohesionada donde los términos 
\textit{speculate}, \textit{grip}, \textit{recruit}, \textit{newsman} y \textit{mad} aparecen en 
posiciones destacadas en ambas métricas, al observar los términos que ganan peso (\textit{biology, enrollment,
academic, scholar}), aparece un núcleo temático muy específico relacionado con la
universidad y la investigación con un mismo valor para todos pesos.

\vspace{0.5cm}
\noindent
\begin{minipage}[t]{0.48\textwidth}
\centering
\begin{tabular}{r l r r}
\hline
Id & Label & Grado & \shortstack{Grado \\ con pesos} \\
\hline
5030 & \textcolor{ForestGreen}{recruit} & 17 & 13.0521 \\
2591 & \textcolor{ForestGreen}{newsman} & 16 & 11.2682 \\
2598 & \textcolor{ForestGreen}{speculate} & 14 & 15.1572 \\
2611 & \textcolor{ForestGreen}{grip} & 14 & 15.1572 \\
5228 & leak & 14 & 8.8128 \\
2318 & chore & 13 & 8.0454 \\
3327 & discrepancy & 13 & 3.4763 \\
943 & qualified & 12 & 5.3733 \\
3182 & marvel & 12 & 8.7209 \\
2603 & \textcolor{ForestGreen}{mad} & 11 & 11.0337 \\
\hline
\end{tabular}

\vspace{0.2cm}
\textbf{Top 10 por Grado}
\end{minipage}
\hspace{0.5cm}
\begin{minipage}[t]{0.48\textwidth}
\centering
\begin{tabular}{r l r r}
\hline
Id & Label & Grado & \shortstack{Grado \\ con pesos} \\
\hline
2598 & \textcolor{ForestGreen}{speculate} & 14 & 15.1572 \\
2611 & \textcolor{ForestGreen}{grip} & 14 & 15.1572 \\
5030 & \textcolor{ForestGreen}{recruit} & 17 & 13.0521 \\
4969 & biology & 9 & 12.2457 \\
5006 & enrollment & 9 & 12.2457 \\
5019 & academic & 9 & 12.2457 \\
5020 & scholar & 9 & 12.2457 \\
2591 & \textcolor{ForestGreen}{newsman} & 16 & 11.2682 \\
2603 & \textcolor{ForestGreen}{mad} & 11 & 11.0337 \\
923 & stay & 7 & 10.6072 \\
\hline
\end{tabular}

\vspace{0.2cm}
\textbf{Top 10 por Grado con pesos}
\end{minipage}
\vspace{0.5cm}

Entre las diferencias más relevantes se observa que términos como \textit{leak}, \textit{chore}, 
\textit{discrepancy}, \textit{qualified} y \textit{marvel} aparecen únicamente en el listado por 
grado, lo que indica que se relacionan con muchos términos distintos pero con menor intensidad 
total, actuando como nodos contextuales amplios dentro de la comunidad.

Las nuevas palabras que aparecen en el ranking de grado con peso \textit{biology}, \textit{enrollment}, \textit{academic}, 
\textit{scholar} y \textit{stay} indican que 
mantienen relaciones más intensas aunque con menos conexiones distintas, rasgo característico de 
términos más específicos o dependientes de contextos concretos.

Sin embargo, hay palabras claves como \textit{recruit}, \textit{grip}, \textit{newsman} y 
\textit{speculate} que hacen pensar en reportes y noticias entorno a un sector academico, es por ello que 
\textit{academic} es el término que mejor define la esencia del clúster si
se prioriza un concepto que englobe una temática centrar.

\newpage


\newpage
\section*{Comunidad 15 - Poet}

En la Comunidad 15 se aprecia una estructura dispersa organizada en cuatro grupos. 
En ambas métricas, los términos \textit{conductor}, \textit{wheel} y 
\textit{style} ocupan posiciones centrales, por lo que actúan como ejes 
semánticos de la comunidad.

Un segundo grupo lo forman \textit{poet}, \textit{austere} y 
\textit{roman}, que apunta a un contexto literario o estético. 
Un tercer conjunto, compuesto por \textit{viewer}, \textit{sigh} y 
\textit{meaningless}, introduce una dimensión interpretativa o valorativa. 
Por último, \textit{dull} y \textit{tag} presentan un grado con pesos 
muy similar, lo que indica una intensidad de coaparición comparable.

\vspace{0.5cm}
\noindent
\begin{minipage}[t]{0.48\textwidth}
\centering
\begin{tabular}{r l r r}
\hline
Id & Label & Grado & \shortstack{Grado \\ con pesos} \\
\hline
2126 & \textcolor{ForestGreen}{wheel} & 14 & 10.2449 \\
2602 & \textcolor{ForestGreen}{conductor} & 14 & 12.8794 \\
2669 & red & 12 & 6.3482 \\
2060 & male & 10 & 5.9002 \\
2665 & powerful & 10 & 3.2460 \\
4733 & ancient & 10 & 6.8665 \\
2646 & lifetime & 9 & 4.2030 \\
4152 & \textcolor{ForestGreen}{dull} & 8 & 7.4662 \\
4715 & \textcolor{ForestGreen}{style} & 8 & 10.5212 \\
2586 & tag & 7 & 7.4639 \\
\hline
\end{tabular}

\vspace{0.2cm}
\textbf{Top 10 por Grado}
\end{minipage}
\hspace{0.5cm}
\begin{minipage}[t]{0.48\textwidth}
\centering
\begin{tabular}{r l r r}
\hline
Id & Label & Grado & \shortstack{Grado \\ con pesos} \\
\hline
2602 & \textcolor{ForestGreen}{conductor} & 14 & 12.8794 \\
4715 & \textcolor{ForestGreen}{style} & 8 & 10.5212 \\
2126 & \textcolor{ForestGreen}{wheel} & 14 & 10.2449 \\
4734 & roman & 7 & 8.8502 \\
4765 & poet & 7 & 8.8502 \\
4772 & austere & 7 & 8.8502 \\
2553 & viewer & 6 & 7.7017 \\
2628 & sigh & 6 & 7.7017 \\
2635 & meaningless & 6 & 7.7017 \\
4152 & \textcolor{ForestGreen}{dull} & 8 & 7.4662 \\
\hline
\end{tabular}

\vspace{0.2cm}
\textbf{Top 10 por Grado con pesos}
\end{minipage}
\vspace{0.5cm}

En la comparación entre ambas métricas se observa que términos como 
\textit{red}, \textit{male}, \textit{powerful}, \textit{ancient} y 
\textit{lifetime} aparecen únicamente entre los primeros por grado sin 
pesos. Esto indica que se conectan con muchos nodos distintos, pero con 
una intensidad media menor.

La copresencia de \textit{wheel} y \textit{conductor}, junto con 
\textit{poet}, sugiere una posible referencia a un texto concreto donde 
estos términos coaparecen, como podría ser el poema \textit{Wheels} de 
Jim Daniels. Puede afirmarse que existe una coocurrencia sistemática 
entre estos términos dentro del mismo conjunto documental, lo que 
sugiere una crítica o análisis de su poemario.

Con base en esto, el término que mejor 
representa la identidad semántica de la comunidad es \textit{poet}, ya que 
actúa como núcleo interpretativo del conjunto y da sentido a la descripción 
del uso conjunto de las palabras.

\newpage
\section*{Comunidad 16 - Golfer}

En la Comunidad 16 se observa una estructura temática definida donde términos como 
\textit{slice}, \textit{golfer}, \textit{par}, \textit{fairway}, \textit{tee} y \textit{bunker} 
aparecen concentrados en posiciones relevantes, lo que indica la presencia de un subcampo léxico 
claramente especializado. Estos nodos muestran relaciones intensas entre sí y forman el núcleo 
semántico dominante del subgrafo.

\vspace{0.5cm}
\noindent
\begin{minipage}[t]{0.48\textwidth}
\centering
\begin{tabular}{r l r r}
\hline
Id & Label & Grado & \shortstack{Grado \\ con pesos} \\
\hline
3082 & swim & 26 & 13.1190 \\
2811 & nerve & 25 & 16.0043 \\
2976 & \textcolor{ForestGreen}{setting} & 23 & 16.5333 \\
2977 & \textcolor{ForestGreen}{italian} & 23 & 16.5333 \\
2775 & \textcolor{ForestGreen}{slice} & 21 & 19.6605 \\
2822 & \textcolor{ForestGreen}{soft} & 21 & 19.6605 \\
1867 & \textcolor{ForestGreen}{chapter} & 20 & 16.3190 \\
2731 & amateur & 20 & 11.1253 \\
3044 & candy & 20 & 8.6244 \\
1563 & motor & 19 & 15.1308 \\
\hline
\end{tabular}

\vspace{0.2cm}
\textbf{Top 10 por Grado}
\end{minipage}
\hspace{0.5cm}
\begin{minipage}[t]{0.48\textwidth}
\centering
\begin{tabular}{r l r r}
\hline
Id & Label & Grado & \shortstack{Grado \\ con pesos} \\
\hline
2762 & golfer & 16 & 20.6904 \\
2763 & par & 16 & 20.6904 \\
2772 & fairway & 16 & 20.6904 \\
2818 & tee & 16 & 20.6904 \\
2820 & bunker & 16 & 20.6904 \\
2775 & \textcolor{ForestGreen}{slice} & 21 & 19.6605 \\
2822 & \textcolor{ForestGreen}{soft} & 21 & 19.6605 \\
2976 & \textcolor{ForestGreen}{setting} & 23 & 16.5333 \\
2977 & \textcolor{ForestGreen}{italian} & 23 & 16.5333 \\
1867 & \textcolor{ForestGreen}{chapter} & 20 & 16.3190 \\
\hline
\end{tabular}

\vspace{0.2cm}
\textbf{Top 10 por Grado con pesos}
\end{minipage}
\vspace{0.5cm}

Entre las diferencias más relevantes se observa que términos como \textit{swim}, \textit{nerve}, 
\textit{amateur}, \textit{candy} y \textit{motor} aparecen únicamente en el listado por grado, lo 
que indica que se relacionan con muchos términos distintos pero con menor intensidad total, 
actuando como nodos contextuales generales dentro de la comunidad.

Por el contrario, los términos \textit{golfer}, \textit{par}, \textit{fairway}, \textit{tee} y 
\textit{bunker} solo aparecen en el ranking ponderado, lo que indica que mantienen relaciones muy 
intensas entre sí aunque con menos conexiones distintas, característica propia de vocabulario 
especializado y cohesionado.

El campo semántico está claramente vinculado al ámbito deportivo del golf, donde aparecen términos 
propios de reglas, jugadas y elementos del campo. Por este motivo, el término más representativo es 
\textit{golfer}, ya que describe de forma directa el dominio temático que estructura el subgrafo.

\newpage
\section*{Comunidad 17 - Cocktail}

En la Comunidad 17 se observa una estructura temática coherente donde términos como 
\textit{wear}, \textit{guest}, \textit{chic}, \textit{pink}, \textit{gown}, \textit{cocktail}, 
\textit{invitation} y \textit{orchestra} aparecen concentrados en posiciones relevantes, lo que 
indica la presencia de un subcampo léxico claramente asociado a contextos sociales formales. Estos 
nodos presentan relaciones intensas entre sí y forman el núcleo semántico dominante del subgrafo.

\vspace{0.5cm}
\noindent
\begin{minipage}[t]{0.48\textwidth}
\centering
\begin{tabular}{r l r r}
\hline
Id & Label & Grado & \shortstack{Grado \\ con pesos} \\
\hline
3189 & \textcolor{ForestGreen}{pink} & 40 & 36.1221 \\
2967 & gown & 34 & 30.8768 \\
3023 & entertain & 34 & 30.8768 \\
3054 & \textcolor{ForestGreen}{wear} & 33 & 45.5821 \\
3283 & \textcolor{ForestGreen}{chic} & 31 & 34.8132 \\
2978 & engagement & 30 & 21.7875 \\
4293 & wardrobe & 30 & 24.4597 \\
2947 & buffet & 29 & 19.6070 \\
2939 & publicity & 28 & 29.7137 \\
3097 & \textcolor{ForestGreen}{guest} & 28 & 36.5391 \\
\hline
\end{tabular}

\vspace{0.2cm}
\textbf{Top 10 por Grado}
\end{minipage}
\hspace{0.5cm}
\begin{minipage}[t]{0.48\textwidth}
\centering
\begin{tabular}{r l r r}
\hline
Id & Label & Grado & \shortstack{Grado \\ con pesos} \\
\hline
1516 & cocktail & 26 & 47.4608 \\
3054 & \textcolor{ForestGreen}{wear} & 33 & 45.5821 \\
3280 & lady & 27 & 40.4794 \\
2928 & marriage & 24 & 38.2980 \\
3097 & \textcolor{ForestGreen}{guest} & 28 & 36.5391 \\
3096 & orchestra & 19 & 36.4394 \\
3189 & \textcolor{ForestGreen}{pink} & 40 & 36.1221 \\
2969 & invitation & 19 & 35.9160 \\
3283 & \textcolor{ForestGreen}{chic} & 31 & 34.8132 \\
761 & color & 22 & 34.4726 \\
\hline
\end{tabular}

\vspace{0.2cm}
\textbf{Top 10 por Grado con pesos}
\end{minipage}
\vspace{0.5cm}

Entre las diferencias más relevantes se observa que términos como \textit{entertain}, 
\textit{engagement}, \textit{wardrobe}, \textit{buffet} y \textit{publicity} aparecen únicamente en 
el listado por grado, lo que indica que se relacionan con muchos términos distintos pero con menor 
intensidad total, actuando como nodos contextuales generales dentro de la comunidad.

Por el contrario, palabras como \textit{cocktail}, \textit{lady}, \textit{marriage}, 
\textit{orchestra}, \textit{invitation} y \textit{color} solo aparecen en el ranking ponderado, lo 
que indica que mantienen relaciones más intensas aunque con menos conexiones distintas, rasgo 
característico de vocabulario asociado a situaciones concretas y cohesivas.

El campo semántico está claramente vinculado a contextos sociales formales y celebraciones, donde 
aparecen términos relacionados con vestimenta, asistentes y elementos propios de eventos. Por este 
motivo, el término más representativo es \textit{cocktail}, ya que describe de forma directa el 
dominio temático que estructura el subgrafo.

\newpage
\section*{Comunidad 18 - Ballroom/Designer}

En la Comunidad 18 se observa una estructura temática cohesionada donde los términos 
\textit{fun}, \textit{designer}, \textit{yellow}, \textit{ballroom}, \textit{handsome} y 
\textit{charm} aparecen en posiciones destacadas en ambas métricas, lo que indica que constituyen 
el núcleo estructural y semántico del subgrafo. Estos nodos presentan simultáneamente una alta 
diversidad de conexiones y una intensidad elevada de coaparición, situándose como los principales 
organizadores conceptuales dentro de la red.

\vspace{0.5cm}
\noindent
\begin{minipage}[t]{0.48\textwidth}
\centering
\begin{tabular}{r l r r}
\hline
Id & Label & Grado & \shortstack{Grado \\ con pesos} \\
\hline
3223 & \textcolor{ForestGreen}{fun} & 32 & 31.4950 \\
3016 & \textcolor{ForestGreen}{handsome} & 31 & 26.6047 \\
4279 & \textcolor{ForestGreen}{designer} & 30 & 30.5695 \\
4461 & \textcolor{ForestGreen}{yellow} & 30 & 30.5695 \\
3273 & \textcolor{ForestGreen}{ballroom} & 27 & 26.7968 \\
4301 & \textcolor{ForestGreen}{charm} & 26 & 23.2020 \\
4308 & lively & 25 & 20.3991 \\
4340 & athlete & 25 & 20.3991 \\
3037 & typical & 23 & 12.9775 \\
1531 & love & 22 & 18.8029 \\
\hline
\end{tabular}

\vspace{0.2cm}
\textbf{Top 10 por Grado}
\end{minipage}
\hspace{0.5cm}
\begin{minipage}[t]{0.48\textwidth}
\centering
\begin{tabular}{r l r r}
\hline
Id & Label & Grado & \shortstack{Grado \\ con pesos} \\
\hline
3223 & \textcolor{ForestGreen}{fun} & 32 & 31.4950 \\
4279 & \textcolor{ForestGreen}{designer} & 30 & 30.5695 \\
4461 & \textcolor{ForestGreen}{yellow} & 30 & 30.5695 \\
3273 & \textcolor{ForestGreen}{ballroom} & 27 & 26.7968 \\
3016 & \textcolor{ForestGreen}{handsome} & 31 & 26.6047 \\
4301 & \textcolor{ForestGreen}{charm} & 26 & 23.2020 \\
4287 & showing & 16 & 21.5013 \\
4342 & charming & 16 & 21.5013 \\
4371 & bake & 16 & 21.5013 \\
4408 & decorate & 16 & 21.5013 \\
\hline
\end{tabular}

\vspace{0.2cm}
\textbf{Top 10 por Grado con pesos}
\end{minipage}
\vspace{0.5cm}

Entre las diferencias más relevantes se observa que términos como \textit{lively}, \textit{athlete}, 
\textit{typical} y \textit{love} aparecen únicamente en el listado por grado, lo que indica que se 
relacionan con muchos términos distintos pero con menor intensidad total, actuando como nodos 
contextuales generales dentro de la comunidad.

Por el contrario, palabras como \textit{showing}, \textit{charming}, \textit{bake} y 
\textit{decorate} solo aparecen en el ranking ponderado, lo que indica que mantienen relaciones 
más intensas aunque con menos conexiones distintas, rasgo característico de términos asociados a 
situaciones concretas y cohesivas.

El campo semántico está vinculado a contextos sociales y estéticos donde predominan referencias a 
apariencia, ambiente y actividades recreativas. Por este motivo, los términos más representativos son 
\textit{ballroom} o \textit{designer} porque en ambas métricas confirma que son los pilares 
que sostienen la identidad temática de esta partición del grafo.


\newpage
\section*{Comunidad 19 - Legislative}

En la Comunidad 19 se observa una estructura temática coherente donde los términos 
\textit{majority}, \textit{senate}, \textit{legislative}, \textit{lieutenant}, \textit{underlie}, 
\textit{undermine}, \textit{extension} y \textit{normal} aparecen en posiciones destacadas en 
ambas métricas, lo que indica que constituyen el núcleo estructural y semántico del subgrafo. 
Estos nodos combinan una diversidad apreciable de conexiones con una intensidad de coaparición 
elevada, situándose como los principales organizadores conceptuales dentro de la red.

\vspace{0.5cm}
\noindent
\begin{minipage}[t]{0.48\textwidth}
\centering
\begin{tabular}{r l r r}
\hline
Id & Label & Grado & \shortstack{Grado \\ con pesos} \\
\hline
659 & \textcolor{ForestGreen}{majority} & 18 & 12.5393 \\
1917 & enjoy & 14 & 2.3023 \\
762 & \textcolor{ForestGreen}{underlie} & 13 & 10.2589 \\
764 & \textcolor{ForestGreen}{extension} & 13 & 9.2418 \\
3383 & consist & 12 & 5.2182 \\
658 & \textcolor{ForestGreen}{legislative} & 11 & 10.1177 \\
521 & \textcolor{ForestGreen}{undermine} & 10 & 10.2161 \\
587 & \textcolor{ForestGreen}{senate} & 10 & 10.2161 \\
644 & \textcolor{ForestGreen}{lieutenant} & 10 & 11.7355 \\
701 & \textcolor{ForestGreen}{normal} & 10 & 10.2161 \\
\hline
\end{tabular}

\vspace{0.2cm}
\textbf{Top 10 por Grado}
\end{minipage}
\hspace{0.5cm}
\begin{minipage}[t]{0.48\textwidth}
\centering
\begin{tabular}{r l r r}
\hline
Id & Label & Grado & \shortstack{Grado \\ con pesos} \\
\hline
659 & \textcolor{ForestGreen}{majority} & 18 & 12.5393 \\
726 & associate & 7 & 11.8017 \\
644 & \textcolor{ForestGreen}{lieutenant} & 10 & 11.7355 \\
762 & \textcolor{ForestGreen}{underlie} & 13 & 10.2589 \\
521 & \textcolor{ForestGreen}{undermine} & 10 & 10.2161 \\
587 & \textcolor{ForestGreen}{senate} & 10 & 10.2161 \\
701 & \textcolor{ForestGreen}{normal} & 10 & 10.2161 \\
658 & \textcolor{ForestGreen}{legislative} & 11 & 10.1177 \\
764 & \textcolor{ForestGreen}{extension} & 13 & 9.2418 \\
489 & fiscal & 9 & 9.0952 \\
\hline
\end{tabular}

\vspace{0.2cm}
\textbf{Top 10 por Grado con pesos}
\end{minipage}
\vspace{0.5cm}

Entre las diferencias más relevantes se observa que términos como \textit{enjoy} y 
\textit{consist} aparecen únicamente en el listado por grado, lo que indica que se relacionan con 
muchos términos distintos pero con menor intensidad total, actuando como nodos contextuales 
generales dentro de la comunidad.

Por el contrario, palabras como \textit{associate} y \textit{fiscal} solo aparecen en el ranking 
ponderado, lo que indica que mantienen relaciones más intensas aunque con menos conexiones 
distintas, rasgo característico de términos asociados a contextos más específicos.

El campo semántico está claramente vinculado al ámbito político e institucional, donde aparecen 
conceptos relacionados con órganos legislativos, jerarquías y procesos normativos. Por este 
motivo, el término más representativo es \textit{legislative}, ya que aparece en ambas tablas y resume 
el eje temático que organiza el subgrafo.

\newpage
\section*{Comunidad 20 - Movement}

En la Comunidad 20 se observa una estructura temática heterogénea donde los términos 
\textit{movement}, \textit{book}, \textit{advanced} y \textit{bag} aparecen en posiciones 
destacadas en ambas métricas, lo que indica que constituyen el núcleo estructural y semántico del 
subgrafo. Estos nodos presentan una diversidad apreciable de conexiones junto con una intensidad 
de coaparición relevante, actuando como puntos organizadores dentro de la red.

\vspace{0.5cm}
\noindent
\begin{minipage}[t]{0.48\textwidth}
\centering
\begin{tabular}{r l r r}
\hline
Id & Label & Grado & \shortstack{Grado \\ con pesos} \\
\hline
1466 & govern & 20 & 9.9990 \\
19 & conduct & 19 & 9.0764 \\
3476 & \textcolor{ForestGreen}{advanced} & 19 & 11.8341 \\
3512 & \textcolor{ForestGreen}{bag} & 19 & 11.8341 \\
493 & \textcolor{ForestGreen}{book} & 16 & 13.5660 \\
1477 & \textcolor{ForestGreen}{movement} & 16 & 14.5587 \\
2206 & organized & 16 & 6.6668 \\
4657 & fundamental & 16 & 8.0733 \\
1405 & type & 15 & 9.6044 \\
1622 & division & 15 & 9.5814 \\
\hline
\end{tabular}

\vspace{0.2cm}
\textbf{Top 10 por Grado}
\end{minipage}
\hspace{0.5cm}
\begin{minipage}[t]{0.48\textwidth}
\centering
\begin{tabular}{r l r r}
\hline
Id & Label & Grado & \shortstack{Grado \\ con pesos} \\
\hline
2205 & dinner & 14 & 22.3284 \\
799 & honor & 9 & 20.4638 \\
2161 & father & 13 & 14.6317 \\
1477 & \textcolor{ForestGreen}{movement} & 16 & 14.5587 \\
1350 & headquarters & 12 & 14.2470 \\
493 & \textcolor{ForestGreen}{book} & 16 & 13.5660 \\
3783 & guidance & 13 & 13.5374 \\
4637 & library & 14 & 13.0460 \\
3476 & \textcolor{ForestGreen}{advanced} & 19 & 11.8341 \\
3512 & \textcolor{ForestGreen}{bag} & 19 & 11.8341 \\
\hline
\end{tabular}

\vspace{0.2cm}
\textbf{Top 10 por Grado con pesos}
\end{minipage}
\vspace{0.5cm}

Entre las diferencias más relevantes se observa que términos como \textit{govern}, 
\textit{conduct}, \textit{organized}, \textit{fundamental}, \textit{type} y \textit{division} 
aparecen únicamente en el listado por grado, lo que indica que se relacionan con muchos términos 
distintos pero con menor intensidad total, actuando como nodos contextuales generales dentro de 
la comunidad.

Por el contrario, palabras como \textit{dinner}, \textit{honor}, \textit{father}, 
\textit{headquarters}, \textit{guidance} y \textit{library} solo aparecen en el ranking ponderado, 
lo que indica que mantienen relaciones más intensas aunque con menos conexiones distintas, rasgo 
característico de términos asociados a contextos concretos y cohesivos.

El campo semántico sugiere un entorno conceptual vinculado a organización, actividad colectiva y 
contextos institucionales o sociales donde aparecen referencias a dirección, eventos y espacios 
estructurados. Por este motivo, el término más representativo es \textit{movement}, ya que aparece 
en ambas tablas y resume el eje temático que articula el subgrafo.
\newpage
\section*{Comunidad 21 - Talent}

En la Comunidad 21 se observa una estructura temática definida donde los términos 
\textit{stick}, \textit{talent} y \textit{vow} aparecen en posiciones destacadas en ambas 
métricas, lo que indica que constituyen el núcleo estructural del subgrafo. Sin embargo, el 
conjunto de términos con mayor peso revela un subcampo léxico claramente especializado vinculado 
al ámbito artístico y escénico.

\vspace{0.5cm}
\noindent
\begin{minipage}[t]{0.48\textwidth}
\centering
\begin{tabular}{r l r r}
\hline
Id & Label & Grado & \shortstack{Grado \\ con pesos} \\
\hline
4579 & \textcolor{ForestGreen}{stick} & 24 & 15.5197 \\
2711 & personality & 21 & 14.1118 \\
1091 & fix & 16 & 6.9702 \\
4655 & \textcolor{ForestGreen}{vow} & 16 & 14.3896 \\
1879 & hero & 15 & 10.1835 \\
1642 & trip & 13 & 9.8713 \\
2900 & rare & 11 & 12.6343 \\
816 & summer & 10 & 8.6867 \\
1280 & huge & 10 & 4.5679 \\
2888 & \textcolor{ForestGreen}{talent} & 10 & 14.6705 \\
\hline
\end{tabular}

\vspace{0.2cm}
\textbf{Top 10 por Grado}
\end{minipage}
\hspace{0.5cm}
\begin{minipage}[t]{0.48\textwidth}
\centering
\begin{tabular}{r l r r}
\hline
Id & Label & Grado & \shortstack{Grado \\ con pesos} \\
\hline
4579 & \textcolor{ForestGreen}{stick} & 24 & 15.5197 \\
4568 & instrument & 10 & 15.3037 \\
4599 & skip & 10 & 15.3037 \\
4616 & pianist & 10 & 15.3037 \\
4631 & ballet & 10 & 15.3037 \\
4632 & performer & 10 & 15.3037 \\
4633 & symphony & 10 & 15.3037 \\
4651 & enthrall & 10 & 15.3037 \\
2888 & \textcolor{ForestGreen}{talent} & 10 & 14.6705 \\
4655 & \textcolor{ForestGreen}{vow} & 16 & 14.3896 \\
\hline
\end{tabular}

\vspace{0.2cm}
\textbf{Top 10 por Grado con pesos}
\end{minipage}
\vspace{0.5cm}

Entre las diferencias más relevantes se observa que términos como \textit{personality}, 
\textit{hero}, \textit{trip}, \textit{summer} y \textit{huge} aparecen únicamente en el listado 
por grado, lo que indica que se relacionan con muchos términos distintos pero con menor 
intensidad total, actuando como nodos contextuales generales dentro de la comunidad.

Por el contrario, palabras como \textit{instrument}, \textit{pianist}, \textit{ballet}, 
\textit{symphony} y \textit{enthrall} solo aparecen en el ranking ponderado, lo que indica que 
mantienen relaciones más intensas aunque con menos conexiones distintas, rasgo característico de 
vocabulario especializado y cohesionado.

El campo semántico está claramente vinculado al ámbito artístico y escénico, donde aparecen 
conceptos relacionados con interpretación musical y actuación. Por este motivo, el término más 
representativo es \textit{talent}, ya que aparece dentro de la tabla ponderada y describe de 
forma directa el dominio temático que organiza el subgrafo.

\newpage
\section*{Comunidad 22 - Battle}

En la Comunidad 22 se observa una estructura cohesionada donde los términos 
\textit{battle}, \textit{communism}, \textit{drastic}, \textit{summon}, \textit{pursue} y 
\textit{routine} aparecen en ambas métricas, lo que indica que constituyen el núcleo estructural 
y semántico del subgrafo. Estos nodos combinan diversidad de conexiones con intensidad de 
coaparición, situándose como los elementos más estables de la comunidad.

\vspace{0.5cm}
\noindent
\begin{minipage}[t]{0.48\textwidth}
\centering
\begin{tabular}{r l r r}
\hline
Id & Label & Grado & \shortstack{Grado \\ con pesos} \\
\hline
3309 & \textcolor{ForestGreen}{drastic} & 14 & 8.7519 \\
3337 & \textcolor{ForestGreen}{summon} & 14 & 8.7519 \\
3594 & \textcolor{ForestGreen}{pursue} & 13 & 7.0110 \\
4112 & \textcolor{ForestGreen}{communism} & 12 & 10.9307 \\
4146 & \textcolor{ForestGreen}{routine} & 12 & 5.8626 \\
355 & \textcolor{ForestGreen}{battle} & 11 & 11.8936 \\
2865 & \textcolor{ForestGreen}{trio} & 10 & 9.3312 \\
2909 & \textcolor{ForestGreen}{preach} & 10 & 9.3312 \\
2449 & hip & 9 & 2.9730 \\
3386 & encouraging & 9 & 5.3125 \\
\hline
\end{tabular}

\vspace{0.2cm}
\textbf{Top 10 por Grado}
\end{minipage}
\hspace{0.5cm}
\begin{minipage}[t]{0.48\textwidth}
\centering
\begin{tabular}{r l r r}
\hline
Id & Label & Grado & \shortstack{Grado \\ con pesos} \\
\hline
355 & \textcolor{ForestGreen}{battle} & 11 & 11.8936 \\
4112 & \textcolor{ForestGreen}{communism} & 12 & 10.9307 \\
2865 & \textcolor{ForestGreen}{trio} & 10 & 9.3312 \\
2909 & \textcolor{ForestGreen}{preach} & 10 & 9.3312 \\
3309 & \textcolor{ForestGreen}{drastic} & 14 & 8.7519 \\
3337 & \textcolor{ForestGreen}{summon} & 14 & 8.7519 \\
440 & threat & 3 & 8.6965 \\
4148 & accompany & 5 & 7.1139 \\
3594 & \textcolor{ForestGreen}{pursue} & 13 & 7.0110 \\
4146 & \textcolor{ForestGreen}{routine} & 12 & 5.8626 \\
\hline
\end{tabular}

\vspace{0.2cm}
\textbf{Top 10 por Grado con pesos}
\end{minipage}
\vspace{0.5cm}

Entre las diferencias más relevantes se observa que términos como \textit{hip} y 
\textit{encouraging} aparecen únicamente en el listado por grado, lo que indica que se conectan 
con varios términos distintos pero con menor intensidad total, funcionando como elementos 
contextuales secundarios dentro del subgrafo.

Por el contrario, palabras como \textit{threat} y \textit{accompany} solo aparecen en el ranking 
ponderado, lo que sugiere relaciones más intensas aunque con menos conexiones distintas, rasgo 
propio de vocabulario dependiente de contextos específicos.

El campo semántico está asociado a acciones, conflicto y discurso ideológico, donde aparecen 
términos relacionados con confrontación, convocatoria y posicionamiento político. Por este motivo, 
el término más representativo es \textit{communism}, ya que sintetiza el eje conceptual dominante 
y organiza semánticamente el conjunto de relaciones del subgrafo.

\newpage
\section*{Comunidad 23 - Scientific}

En la Comunidad 23 se observa una estructura relativamente estable donde los términos 
\textit{implement}, \textit{significance}, \textit{scientific} y \textit{organ} aparecen en ambas 
métricas, lo que indica que constituyen el núcleo estructural y semántico del subgrafo. Estos 
nodos combinan una diversidad notable de conexiones con una intensidad alta de coaparición, por 
lo que actúan como puntos centrales dentro de la red léxica.

\vspace{0.5cm}
\noindent
\begin{minipage}[t]{0.48\textwidth}
\centering
\begin{tabular}{r l r r}
\hline
Id & Label & Grado & \shortstack{Grado \\ con pesos} \\
\hline
840 & \textcolor{ForestGreen}{significance} & 19 & 15.9856 \\
3903 & \textcolor{ForestGreen}{scientific} & 16 & 13.2788 \\
4206 & \textcolor{ForestGreen}{implement} & 16 & 16.1372 \\
2234 & nominate & 15 & 7.6020 \\
2578 & \textcolor{ForestGreen}{organ} & 15 & 13.1642 \\
3428 & suburban & 15 & 11.4732 \\
1367 & enemy & 14 & 7.2114 \\
2954 & success & 13 & 9.2594 \\
787 & client & 12 & 4.8045 \\
1411 & grocery & 12 & 7.1645 \\
\hline
\end{tabular}

\vspace{0.2cm}
\textbf{Top 10 por Grado}
\end{minipage}
\hspace{0.5cm}
\begin{minipage}[t]{0.48\textwidth}
\centering
\begin{tabular}{r l r r}
\hline
Id & Label & Grado & \shortstack{Grado \\ con pesos} \\
\hline
4206 & \textcolor{ForestGreen}{implement} & 16 & 16.1372 \\
840 & \textcolor{ForestGreen}{significance} & 19 & 15.9856 \\
3903 & \textcolor{ForestGreen}{scientific} & 16 & 13.2788 \\
2578 & \textcolor{ForestGreen}{organ} & 15 & 13.1642 \\
2653 & arrangement & 9 & 12.8308 \\
4904 & informal & 9 & 12.6511 \\
4916 & illusion & 9 & 12.6511 \\
4923 & broad & 9 & 12.6511 \\
4932 & fatal & 9 & 12.6511 \\
4955 & initiative & 9 & 12.6511 \\
\hline
\end{tabular}

\vspace{0.2cm}
\textbf{Top 10 por Grado con pesos}
\end{minipage}
\vspace{0.5cm}

Entre las diferencias más relevantes se observa que términos como \textit{nominate}, 
\textit{suburban}, \textit{enemy}, \textit{client} y \textit{grocery} aparecen únicamente en el 
listado por grado, lo que indica que se conectan con múltiples palabras distintas pero con menor 
frecuencia total, funcionando como nodos contextuales amplios dentro del subgrafo.

Por el contrario, palabras como \textit{arrangement}, \textit{informal}, \textit{illusion}, 
\textit{initiative} o \textit{fatal} solo aparecen en el ranking ponderado, lo que sugiere que 
mantienen relaciones más intensas aunque con menos conexiones distintas, rasgo típico de términos 
más especializados y dependientes de contextos concretos.

El campo semántico está asociado a procesos conceptuales y terminología analítica vinculada al 
ámbito científico y metodológico. Por este motivo, el término más representativo es 
\textit{scientific}, ya que sintetiza el eje temático dominante y organiza semánticamente el 
conjunto de relaciones del subgrafo.

\newpage
\section*{Comunidad 24 - Murder}

En la Comunidad 24 se observa una estructura donde los términos \textit{murder}, \textit{death}, 
\textit{delegate}, \textit{grade} y \textit{uncle} aparecen en ambas métricas, lo que indica que 
constituyen el núcleo estructural del subgrafo. Estos nodos presentan simultáneamente diversidad 
de conexiones e intensidad de coaparición, por lo que actúan como puntos estables dentro de la 
red léxica.

\vspace{0.5cm}
\noindent
\begin{minipage}[t]{0.48\textwidth}
\centering
\begin{tabular}{r l r r}
\hline
Id & Label & Grado & \shortstack{Grado \\ con pesos} \\
\hline
3361 & \textcolor{ForestGreen}{murder} & 18 & 13.8694 \\
2247 & mature & 17 & 10.6203 \\
171 & elaborate & 16 & 11.1266 \\
444 & \textcolor{ForestGreen}{death} & 16 & 13.1834 \\
2210 & \textcolor{ForestGreen}{delegate} & 16 & 11.4510 \\
2994 & hint & 16 & 6.3101 \\
3640 & \textcolor{ForestGreen}{grade} & 16 & 11.2311 \\
2942 & \textcolor{ForestGreen}{uncle} & 15 & 11.1851 \\
2943 & aunt & 15 & 11.1851 \\
3116 & civilian & 15 & 6.1706 \\
\hline
\end{tabular}

\vspace{0.2cm}
\textbf{Top 10 por Grado}
\end{minipage}
\hspace{0.5cm}
\begin{minipage}[t]{0.48\textwidth}
\centering
\begin{tabular}{r l r r}
\hline
Id & Label & Grado & \shortstack{Grado \\ con pesos} \\
\hline
3361 & \textcolor{ForestGreen}{murder} & 18 & 13.8694 \\
444 & \textcolor{ForestGreen}{death} & 16 & 13.1834 \\
3582 & collection & 12 & 12.2605 \\
3520 & appeal & 10 & 11.8602 \\
3636 & institution & 13 & 11.5116 \\
3662 & existence & 13 & 11.5116 \\
2210 & \textcolor{ForestGreen}{delegate} & 16 & 11.4510 \\
3583 & faculty & 12 & 11.2654 \\
3640 & \textcolor{ForestGreen}{grade} & 16 & 11.2311 \\
2942 & \textcolor{ForestGreen}{uncle} & 15 & 11.1851 \\
\hline
\end{tabular}

\vspace{0.2cm}
\textbf{Top 10 por Grado con pesos}
\end{minipage}
\vspace{0.5cm}

Entre las diferencias más relevantes se observa que términos como \textit{mature}, 
\textit{elaborate}, \textit{hint}, \textit{aunt} y \textit{civilian} aparecen únicamente en el 
listado por grado, lo que indica que se relacionan con varios términos distintos pero con menor 
frecuencia total, funcionando como elementos contextuales dentro de la comunidad.

Por el contrario, palabras como \textit{collection}, \textit{appeal}, \textit{institution}, 
\textit{existence} o \textit{faculty} solo aparecen en el ranking ponderado, lo que indica que 
mantienen relaciones más intensas aunque con menos conexiones distintas, rasgo propio de términos 
más específicos o dependientes de contextos concretos.

El campo semántico se organiza en torno a conceptos asociados a mortalidad, consecuencias y 
contextos sociales relacionados con fallecimiento y responsabilidad. Por este motivo, el término 
más representativo es \textit{death}, ya que sintetiza el eje temático dominante y estructura el 
significado global del subgrafo.

\newpage

\section*{Comunidad 25 - Enact}

En la Comunidad 25 se observa una estructura donde el término \textit{enact} aparece en ambas 
métricas junto con varios nodos de fuerte carga jurídica y normativa, lo que indica que el 
subgrafo se organiza en torno a procesos legales y regulatorios. La coexistencia de términos 
relacionados con delito, sentencia y legislación muestra que el núcleo semántico está definido 
por vocabulario propio del ámbito judicial.

\vspace{0.5cm}
\noindent
\begin{minipage}[t]{0.48\textwidth}
\centering
\begin{tabular}{r l r r}
\hline
Id & Label & Grado & \shortstack{Grado \\ con pesos} \\
\hline
529 & \textcolor{ForestGreen}{enact} & 26 & 20.7854 \\
894 & employer & 26 & 14.4611 \\
778 & criminal & 25 & 8.9822 \\
3467 & missile & 24 & 10.8210 \\
704 & credit & 22 & 13.0036 \\
779 & disclosure & 21 & 11.5444 \\
914 & excess & 21 & 12.1162 \\
936 & freedom & 21 & 14.4427 \\
3491 & indictment & 20 & 14.2949 \\
3506 & convict & 20 & 14.2949 \\
\hline
\end{tabular}

\vspace{0.2cm}
\textbf{Top 10 por Grado}
\end{minipage}
\hspace{0.5cm}
\begin{minipage}[t]{0.48\textwidth}
\centering
\begin{tabular}{r l r r}
\hline
Id & Label & Grado & \shortstack{Grado \\ con pesos} \\
\hline
2074 & sentence & 14 & 22.9521 \\
529 & \textcolor{ForestGreen}{enact} & 26 & 20.7854 \\
1362 & vehicle & 18 & 17.9736 \\
153 & citizen & 14 & 17.9622 \\
2934 & sister & 15 & 16.2134 \\
3882 & interfere & 18 & 16.0827 \\
3091 & die & 13 & 15.7493 \\
837 & prosecution & 12 & 15.6911 \\
838 & unfair & 12 & 15.6911 \\
855 & constitute & 12 & 15.6911 \\
\hline
\end{tabular}

\vspace{0.2cm}
\textbf{Top 10 por Grado con pesos}
\end{minipage}
\vspace{0.5cm}

Entre las diferencias más relevantes se observa que términos como \textit{employer}, 
\textit{missile}, \textit{credit}, \textit{disclosure} o \textit{excess} aparecen únicamente en el 
listado por grado, lo que indica que se relacionan con muchos términos distintos pero con menor 
intensidad total, funcionando como nodos contextuales amplios dentro del subgrafo.

Por el contrario, palabras como \textit{sentence}, \textit{prosecution}, \textit{constitute} o 
\textit{unfair} solo aparecen en el ranking ponderado, lo que sugiere relaciones más intensas 
aunque con menos conexiones distintas, rasgo característico de vocabulario más específico y 
dependiente de contextos jurídicos concretos.

El campo semántico está vinculado a procedimientos legales, responsabilidad penal y aplicación de 
normas. Por este motivo, el término más representativo es \textit{criminal}, ya que sintetiza el 
dominio conceptual predominante y organiza semánticamente el conjunto de relaciones del subgrafo.

\newpage
\section*{Conclusiones}

 
\newpage


\begin{thebibliography}{1}

    \bibitem{nlp_nltk_to_HuggingFace}
    Elavarasan P.
    \textit{Natural Language processing (Basics to SOTA models) Part-1}.
    Disponible en \url{https://digitaldata.science.blog/2022/01/18/natural-language-processing-basics-to-sota-models-part-1/}.

    \bibitem{nltk_wikipedia}
    Wikipedia.
    \textit{Natural Language Toolkit}.
    Disponible en \url{https://en.wikipedia.org/wiki/Natural_Language_Toolkit}.
        
    \bibitem{nltk_official}
    NLTK Team.
    \textit{Natural Language Toolkit (NLTK) Documentation}. 
    Disponible en \url{https://www.nltk.org/}. 2026.

    \bibitem{nltk_first_steps}
    Luis Merino Ulizarna.
    \textit{NLTK: tus primeros pasos con Procesamiento del Lenguaje Natural}. 
    Disponible en \url{https://adictosaltrabajo.com/2023/07/27/nltk-python/}. 2026.

    \bibitem{spacy_wikipedia}
    Wikipedia.
    \textit{spaCy}.
    Disponible en \url{https://en.wikipedia.org/wiki/SpaCy}.
    
    \bibitem{spacy_api}
    spaCy Developers (Explosion AI).
    \textit{spaCy: Library Architecture}. 
    Disponible en \url{https://spacy.io/api}. 2025.

    \bibitem{spacy_architecture}
    Explosion AI.
    \textit{spaCy: Model Architectures}.
    Disponible en \url{https://spacy.io/api/architectures}.

    \bibitem{bert_wikipedia}
    Wikipedia.
    \textit{BERT (modelo de lenguaje)}.
    Disponible en \url{https://es.wikipedia.org/wiki/BERT_(modelo_de_lenguaje)}.

    \bibitem{bert_ner}
    Marcello Politi (2022).
    \textit{Custom Named Entity Recognition with BERT}.
    Disponible en \url{https://towardsdatascience.com/custom-named-entity-recognition-with-bert-cf1fd4510804/}.

    \bibitem{ner_survey}
    Vikas Yadav and Steven Bethard (2019).
    \textit{A Survey on Recent Advances in Named Entity Recognition}. 
    Disponible en \url{https://arxiv.org/abs/1910.11470}.

    \bibitem{conll2003deepai}
    DeepAI. (s.f.).
    \textit{CoNLL-2003 Named Entity Recognition Dataset}.
    Disponible en \url{https://data.deepai.org/conll2003.zip}.

\end{thebibliography}



\end{document}


